\chapter{Summary \& Conclusions}
\lhead{Chapter 5 \emph{Summary \& Conclusions}}
\label{chap:5}
%\autoref{chap:5}

% I should also mention about the further work that can be done or in a seperate chapter "Outlook" !
--  Suggest possibilities for further developments and tests
--  Present other remarks for future work

-- Outlook:
The application can be further expanded in order to offer more capabilities. Practically: The heat map that was presented in the first part of the application, will also have a 3rd dimension, in which it will show the time that the satellite has passed from every spot of the Earth. The benefit of that will be that when the user will specify a certain area of interest then, the orbits of various satellites can be calculated and then based on the time that each satellite passes, it can be found whether there are some common periods that the satellites pass from the same area at the same (or almost the same) time. 
(So far -in the first part of the application- there is only a count when a satellite passes from an area. In this aforementioned idea, the exact time will be saved.)
This idea came from the Cryo2Ice mission * Find source :https://www.bbc.com/news/science-environment-53326490