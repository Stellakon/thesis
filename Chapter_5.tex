\chapter{Discussion, future work \& conclusions}
\lhead{Chapter 5 \emph{Discussion, future work \& conclusions}}
\label{chap:5}
%\autoref{chap:5}



% So, maybe another chapter will be for "Evaluation and Discussion" ??
%The discussion will consist of argumentation. In other words, you investigate a phenomenon from several different perspectives. To discuss means to question your findings, and to consider different interpretations. Here are a few examples of formulations that signal argumentation:
%
%On the one hand … and on the other …
%However …
%… it could also be argued that …
%… another possible explanation may be …

% In the evaluation part:
%-- Judge your results critically
%-- Compare your results with the results form other methods/works/experiments (keyword: validation)
%-- show problems

%- Strengths, limitations
%- Importance of thesis contribution

% I will test the software with the big and well-known satellites, like Sentinel-1. 
% I should also mention about the further work that can be done or in a seperate chapter "Outlook" !
%--  Suggest possibilities for further developments and tests
%--  Present other remarks for future work: prioritization of the missions based on the added value in the society. So say that this work has helped into this direction...
%* Info about the tilt angle
%* Extend to other fields - Communication etc
%++ what can be done more? Finally, more complex sensor shapes could be investigated and dierent projection geometries could be studied to ensure accuracy of the method at higher target latitudes.
%++ Calculation of revisit time metrics considering
%time of night/day or lighting conditions would also enable
%particular application for optical Earth observation
%sensors. (mallon min to anafereis. Einai polu sxetiko me tin ergasia... Mipos na to eixa kanei?)

%-- Outlook:
%The application can be further expanded in order to offer more capabilities. Practically: The heat map that was presented in the first part of the application, will also have a 3rd dimension, in which it will show the time that the satellite has passed from every spot of the Earth. The benefit of that will be that when the user will specify a certain area of interest then, the orbits of various satellites can be calculated and then based on the time that each satellite passes, it can be found whether there are some common periods that the satellites pass from the same area at the same (or almost the same) time. 
%(So far -in the first part of the application- there is only a count when a satellite passes from an area. In this aforementioned idea, the exact time will be saved.)
%This idea came from the Cryo2Ice mission %* Find source :https://www.bbc.com/news/science-environment-53326490

% !! Parallel computing methods can be used to improve this performance significantly by increasing the number of computations which can be performed in a given period of time.

%Who will be helped with this work? It will give the incentives etc etc buuuut These plots on their own and the supporting data can provide valuable reference for Earth observation system designers in the early phases of a mission development lifecycle. (By Crisp_2018 p.10) 