\chapter{Methodology}
\lhead{Chapter 3 \emph{Methodology}}
\label{chap:3}
%\autoref{chap:3}

% How I did it? First part of application by having the keplerian elements ....
% The implementation was done using Python and PostgreSQL (?)
%Print earth map. Based on the FOV and the revisit frequency, I can see how the Earth will be printed/colored (heat map). So, we want to track coverage/ revisit time across the equator.

% Tip: "Do not present data at this stage but use sketches or synthetic data instead"

% The app is used for analysis. It is not opeartional (obviously), so no need to check for mistakes in the import of the parameters.

% In the part of the database: mention that we focus on LEO and EO satellites!

% Define revisit time --> average (median) revisit time in equator

\section{Sources of data}
\bigskip

\section{Data gathering procedure}
\bigskip

\section{Data analysis}
\bigskip

%Add a chapter "Results"???

%In the chapter, where you will talk about the database, add big tables about the results - with the commercial and non-commercial constellations & then put a reference next to the other table that I have only the commercial EO satellites in the "Chapter_1" (~ Seira 234)!
