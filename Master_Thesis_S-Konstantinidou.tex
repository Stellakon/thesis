
\documentclass[12pt,a4paper,notitlepage,oneside,openright]{report}
%\documentclass[12pt,a4paper,draft,notitlepage,twoside,openright]{report}

%Libraries for greek
\usepackage{xltxtra}
%\usepackage{xgreek} 
\setmainfont[Mapping=tex-text]{Times New Roman}
%\usepackage{cmap}
%\usepackage{ucs}
%\usepackage{kerkis}
%\usepackage[english,greek]{babel}
\usepackage[utf8x]{inputenc}
\usepackage{multirow}

\usepackage[colorlinks=true]{hyperref} 					%hyperlinks
\hypersetup{linkcolor=blue,citecolor=blue,linktoc=page} %hyperlink options
\usepackage[a4paper,top=3cm,bottom=2cm,left=3cm,right=3cm,marginparwidth=1.75cm]{geometry}
\usepackage{setspace}
\onehalfspacing
\raggedbottom % disables vertical alignment & stretching
\setlength\parindent{0pt} %me auto den exw esoxes se kamia paragrafo (diladi noindent pantou)
\usepackage[parfill]{parskip}
\usepackage{titlesec}
\titleformat{\chapter}[display]{\normalfont\Huge\bfseries}{\chaptertitlename\ \thechapter}{15pt}{\Huge}
\titlespacing{\chapter}{0pt}{40pt}{21pt}
\titlespacing{\section}{0pt}{0pt}{0pt}
\titlespacing{\subsection}{0pt}{0pt}{0pt}
\titlespacing{\subsubsection}{0pt}{0pt}{0pt}
\usepackage{tocloft}
%\recommand\cftchapafterpnum{\vskip5pt}

\usepackage{fancyhdr} %control of page headers and footers- me auto vazw tis eutheies grammes panw kai katw stis selides
\renewcommand{\footrulewidth}{0.4pt}
%\lhead[\rm \bf \thepage]{\fancyplain{}{\sl{\rightmark}}}
\rhead{\rm \bf \thepage}{\fancyplain{}{\sl{\leftmark }}}
\chead{}\lfoot{}\rfoot{}\cfoot{}

\usepackage{graphicx}
\usepackage{amsmath} %extra math
\usepackage{mathtools}
\usepackage{amssymb} %extra symbols
\usepackage{gensymb}
\usepackage{breqn}  %break equations
\usepackage[notlof,notlot,nottoc]{tocbibind} %bibliography is not a numbered chapter
\usepackage{fixltx2e} %gia na grafw textsuperscript + textsubscript
\usepackage{enumitem} %gia to itemize
\usepackage{relsize} %gia \mathlarger + \mathsmaller
\usepackage{subcaption}
\usepackage{mwe}
\usepackage{siunitx} %gia na xrisimopoiisw to \ang{360}--> gia na valei to symvolo twn moirwn (degree)
\usepackage{textcomp}

\usepackage{algorithm}
\usepackage[noend]{algpseudocode} %gia pseudokwdika- otan thelw na perigrapsw kai kala code

\makeatletter
\def\BState{\State\hskip-\ALG@thistlm}
\makeatother

\usepackage{amssymb}
\usepackage{longtable}

\usepackage{makecell}

\usepackage{xr}
\usepackage{adjustbox} %auto prepei na kathorizei kai an einai i current page odd or even
\usepackage[margin=4pt,font=normalsize,labelfont=bf,textfont=it]{caption}
%\usepackage{subcaption}
%\usepackage{sidecap}
\usepackage{float}
\usepackage{ragged2e} %gia na kanei fully justified text

\newcommand{\HRule}{\rule{\linewidth}{0.5mm}} % New command to make the lines in the title page

\usepackage{color} 								%hilight pachage
\usepackage[dvipsnames]{xcolor}
\newcommand{\hilight}[1]{\colorbox{yellow}{#1}} %hilight color

\usepackage[hang,flushmargin]{footmisc} %a range of footnote options - noindent
\usepackage[normalem]{ulem} %Gia na kanw underline kapoio text
\usepackage{esvect} %For vector with dot

\renewcommand{\figurename}{Figure} %Ti leei otan vazw eikones (Fig.1 ...)
\renewcommand{\contentsname}{Table of Contents}
\renewcommand{\chaptername}{Chapter}
\renewcommand{\tablename}{Tab.} %Ti leei otan vazw tables (Capt.1 ...)

%% FOR MATLAB CODE
\usepackage{listings}
\usepackage{color} %red, green, blue, yellow, cyan, magenta, black, white
\definecolor{mygreen}{RGB}{28,172,0} % color values Red, Green, Blue
\definecolor{mylilas}{RGB}{170,55,241}

\renewcommand\bibname{References} %If I don't use it, then instead of the name "References", it will say "Bibliography"

%Package for list of acronym in the end of the document
\usepackage{enumitem}
\newlist{abbrv}{itemize}{1}
\setlist[abbrv,1]{label=,labelwidth=1in,align=parleft,itemsep=0.1\baselineskip,leftmargin=!}

\usepackage{flafter} % Not to let the images and the stuff be presented in the document before their actual reference


\begin{document}
\pagenumbering{roman} \pagestyle{plain}
\thispagestyle{empty}
\begin{minipage}{0.2\textwidth}
\includegraphics[width=\textwidth]{./Images/ESA-logo-and-wordmark.png}
\end{minipage}
\begin{minipage}[r]{0.5\textwidth}
\begin{center}
\hspace*{0.5cm}  \Large {\color{Violet}    }\large \\
\hspace*{0.5cm}  {\color{Violet}       } \\
\end{center}
\end{minipage}
\begin{minipage}[r]{0.2\textwidth}
\includegraphics[width=\textwidth]{./Images/TU-Muenchen.png}
\end{minipage}
\begin{minipage}[r]{0.1\textwidth}
\includegraphics[scale=0.13]{./Images/Department_logo.png}
\end{minipage}

\bigskip
\begin{minipage}{0.62\textwidth}
\begin{flushleft}
\normalsize{European Space Agency \\European Space Operations Centre (ESOC) \\Space Debris Office}
\end{flushleft}
\end{minipage}
\begin{minipage}{0.37\textwidth}
\begin{flushleft}
\normalsize{Technische Universität München \\Fakultät für Luftfahrt, Raumfahrt und Geodäsie \\Lehrstuhl für Astronomische und Physikalische Geodäsie }
%\\Univ.-Prof. Dr.techn. Mag.rer.nat. Roland Pail
\end{flushleft}
\end{minipage}

\vspace{3cm}
\begin{center}
\Large{\textbf{Addressing space infrastructure needs and utility on the example of Earth observation satellites \\- \\}}
\Large{\textbf{Potential ways to reduce space traffic by sharing resources}}
% or --> Potential ways to reduce space traffic - Addressing space infrastructure needs and utility on the example of Earth observation satellites
% or --> Addressing space infrastructure needs and utility - potential ways to reduce space traffic on the example of Earth observation satellites
\end{center}
\vspace{2.1cm}
\begin{center}
\large{Master Thesis of:\\
\textit{\textbf{Styliani Konstantinidou}} \\}
\vspace{1cm}
\large{Master's Course in Earth Oriented Space Science and Technology}
\end{center}

\vspace{2.5cm}
\begin{flushleft}
\large{Supervisors: Dr. Vitali Braun \\ \hspace*{2.7cm}Prof. Urs Hugentobler}
\end{flushleft}

\vfill
\begin{center}
\normalsize Munich, 2020
\end{center}



%Abstract
\newpage \mbox{} %mbox-> creates a box just wide enoigh to hold the text in his arguments
\newpage \mbox{}
\thispagestyle{empty}
%\vspace{\fill}
\begin{center}
\Large{\textbf{Abstract}} 
\end{center}
\bigskip
\par 	
The motives of this thesis were the growing numbers of objects in space and the potential to exceed the carrying capacity in the near-Earth orbits. These are addressed by running a first experiment with the software written during this thesis, which can give the incentives to create space traffic management rules based on the orbit and capacity allocation, as well as the added value that each new satellite may have. An attempt to approach its development is to focus on Earth Observation (EO) satellites at LEO. Some capabilities of the software are the calculation of the revisit time of a single satellite or of a constellation at the equator or at a latitude of interest to the user, and the investigation of the orbital characteristics of a satellite in order to achieve a more frequent revisit time together with a group of satellites. Overall, those satellites, which can achieve a significant increase of the revisit rate, have a higher added value compared to other satellites/ mission, which do not come with any considerable changes. The creation of a database of EO satellites connected to the software increased the capabilities of the software, since a user can also find out whether there are operational satellites offering a specific service. For the development of the software, the necessary input data are the TLE set of the satellite, as well as the parameters related to the satellite’s EO sensor, such as the swath width, operational lifetime, spatial resolution and other. Having these parameters, the position of the sub-satellite’s points on Earth is found and based on the swath width the revisit time of the satellite is calculated. The experiments and results showed that higher revisit rates can be achieved when adding a satellite in the same orbital altitude and not placing it in a higher or lower one. Moreover, the addition of a single satellite to an already large constellation does not result in a significant reduction of the revisit time.

Keywords: revisit time, database, classification, added-value, Earth Observation satellites, LEO

%Abstract should give the “Big Picture” of your work
%Should inform the reader (say in 300 words) about: the goal of the work, applied methodology, result of the work
% 

%---------------------------------------------------------------------------
%Acknowledgments
%\newpage \mbox{}
\newpage \mbox{}
\thispagestyle{empty}
%\vspace{\fill}
\Large{\textbf{Acknowledgments}} 
\par
\bigskip
\normalsize I would like to thank Dr. Vitali Braun for having included me in this exciting topic and for his thoughtful supervision. Despite this unprecedented and unstable period, his support was valuable, which kept my interest undimmed and his guidance is truly appreciated.

Additionally, I am also thankful to Prof. Urs Hugentobler for his help and his precious advice. The received feedback was always immediate and constructive.

Finally, my thoughts go to my family and friends, for their much appreciated affection and love.


\vspace{\fill} %\vspace{\fill} in a paragraph will add the filling vertical space below the line in which it eventually appears;

%Declarations
\newpage \mbox{}
\renewcommand{\abstractname}{\Large{{ }}}
\begin{abstract}
\vspace{14cm}
{\it
\noindent
This thesis is a presentation of my original research work. Wherever contributions of others are involved, every effort is made to indicate this clearly, with due reference to the literature, and acknowledgement of collaborative research and discussions.}
\vspace{2cm}
\begin{center}
{\it 
Munich, December 22, 2020}
\end{center}
\begin{center}
\vspace{0.5cm}
{\it
Styliani Konstantinidou \vspace{0.3cm}
\begin{figure}[h]
\centering
\includegraphics[width=0.3\textwidth]{Images/23600678_10155917223598909_906035917_o.png}
\end{figure}}
\end{center}
\end{abstract}



\newpage \mbox{}
\tableofcontents

\newpage \mbox{}
\listoffigures
%\addcontentsline{toc}{chapter}{List of figures} %In order to add it on the "Table of Contents"

\newpage \mbox{}
\listoftables

\clearpage \pagenumbering{arabic} \pagestyle{fancy} %to fancy pagestyle einai auto pou vazei tis eutheies grammes panw kai katw apo tis selides!
\chapter{Introduction}
\lhead{Chapter 1 \emph{Introduction}}
\label{chap:1}
%\autoref{chap:1}

\section{Motivation}
\label{chap:1.1}
\bigskip
\bigskip

In 1957 the first satellite in history was launched and since then thousands other objects were placed into orbit. \cite{Belward 2015}, \cite{ESA 2020} Two decades later NASA’s scientist Kessler proposed a scenario, in which he theorized that the increasing density of the objects in the Low Earth Orbit (LEO) cause collisions and the generated space debris increase even more the probability of further collisions. The humanity today is confronted more than ever with the realization of this theory, which has gone down as Kessler Syndrome.

There have been taken many important steps towards raising the consciousness about the overpopulation problem, as well as the enactment of rules regarding the capacity and orbit allocation. However, despite the fact that coordination committees and working groups have been established and they are involved in the mitigation guidelines, the existed space traffic management rules deal today only with the frequency allocation used by satellites. \cite{Griffin} %"First come, first served" procedure

Having said that, the motivation of this thesis is the creation of an application, which will help into supplementing the space traffic management rules regarding the physical location of satellites based on the added value that each satellite contributes to the society in terms of additional services or data. The focus of this application is on Earth Observation (EO) satellites at LEO, which is one attempt to approach and tackle the problem. The creation of a database, which hosts information about the EO satellites that are currently on orbit at the LEO region, and the connection of it to the application offers additional services. The user can request and find out whether there are existing orbiting and operational EO satellites or constellations that produce EO product with a certain characteristics. In this way, not only it can be found what is the current state of the operational satellites, which have common attributes, %and thus conclude how many of the already existing satellites
but also it can give the incentives to business entities %private and public companies or institutes
to cooperate, offer and share the data acquired from their already orbiting satellites with other organizations that seek or are interested of those. This can be a scenario beneficial for both sides, while at the same time it helps towards conserving the common space resources of LEO, which have been squandered over the past years.  

\bigskip
\bigskip

\newpage
\section{Scope and key assumptions}
\label{chap:scope}
\bigskip

%\textit{Edit it...}

The first part of this master thesis pertains to the Theoretical Framework of the discussed problem; the overpopulation. The points being addressed are: the problem of overpopulation, the concept of orbital capacity, the state of LEO and its critical regions, the basic characteristics of LEO, the different ways of dealing with the problem, as well as the mitigation guidelines, which have been already put into action.

The second part is dedicated to the development of an application, which can give the incentives to create space traffic management rules based on the orbit and capacity allocation, as well as the added value that each new satellite has. More specifically, the concept of added value is based on the idea that each new satellite is prioritized to be launched when it provides some additional services or data to the society, which are not offered from another existing and functional satellite/ constellation.
%It is not monetary.

An attempt to approach the development of such application is to focus on Earth Observation (EO) satellites at LEO. The reason behind the selection of EO as the first group of satellites to be tested in the application, is because the calculation of the capabilities of an EO satellite is more apparent in comparison to other scientific or technological missions. In the case of telecommunication missions, the estimation of the added value that each satellite provides to the society is more complex, due to the fact that there is a competition between the satellite and terrestrial industry. All things considered, this application focus on EO satellites in LEO and has three main capabilities.
%One should also calculate the bandwidths, apart from the coverage, in the case of the Communication Satellites.

\begin{itemize}
\item The first capability is that the user can define a satellite mission or constellation, which means that the number of satellites, the altitude, the orbital plane, resolution etc. is known. Then, the software can answer what is the revisit time of that defined mission in the equator or in specified latitude that the user is interested of.
\item The second capability is the following: The characteristics of an EO product or/and key attributes that an EO satellite has, are defined. As an example, it can be products from passive imagery sensors, which have a certain revisit time and resolution. As a result, the software can provide to the user the orbital characteristics of the satellite or constellation that can supply the desired data.
\end{itemize}

An extension of this application and its capabilities is the creation of a database of Earth Observation satellites.

\begin{itemize}
\item In both mentioned capabilities of the application, the user can find out, if there are currently satellites which offer the desired data that they had asked for and if such satellites exist, to name them.
\end{itemize}

The database contributes greatly to the creation of a well-rounded application, which can help into mitigating the problem of overpopulation. Since the user can find out about whether existing satellites do a certain/asked project or offer a specific service, thus he/she can decide whether a new satellite is needed to be launched based on whether the desired service is offered or not by someone else. In this way, it can also incentivize and encourage the future companies to cooperate with each other, instead of constructing/ designing/ launching a new constellation. In a more detailed way, such a cooperation between different entities would mean that not only the data taken by the satellites could be shared between the collaborators, but also the companies would also be interested in placing their platform/ sensors to an existing constellation operated by another company. As a result, not only data/ images are shared, but also the orbit itself and its characteristics such as its altitude, which is usually translated in terms of resolution in the field of EO. All in all, a rule in the field of space traffic management can be formulated based on the added value that a new satellite offers.
%For example, if you define a satellite and this exists in the database, then the value for competition is high, but there is no value for the environment.

The concept of prioritization of the missions based on the additional services that they offer before being launched seems very promising. So, one objective of this thesis is to create awareness among the space industry and the society. Another one is the creation of an application, which it can contribute to the formation of legislation based on the added value of the satellites. Hence, consensus in the corresponding space traffic management committees can be reached and the mitigation of the problem of overpopulation can be actualized by attenuating the number of satellites that are launched.


%A space traffic management application towards the estimation of added value.

%Lastly, the part of the Evaluation follows, as well as parts of Summary and Conclusions and that of Future work.


\bigskip
\section{Current launching trend}
\bigskip

\subsection{Active commercial space industry}
\bigskip
%Source: Space Environment Statistics \& the latest ESA's Environment Report with graphs and numbers (https://sdup.esoc.esa.int/discosweb/statistics/)

%LEO and compare with GEO
The world goes in the direction of launching mainly small satellites especially in LEO and more specifically in large constellations. As can be seen in Figure \ref{launch_traffic_LEO}, in a period of 10 years, namely since 2010, a fourfold increase occurred in the number of objects that were launched into LEO. On the other hand, the number of the launched objects in GEO has an upward trend without showing any further sharp peaks (Figure \ref{launch_traffic_GEO}). It is also worth mentioning that in both cases the majority of the objects placed into orbit have a commercial funding source.

\begin{figure}
\centering
\includegraphics[width=0.9\textwidth]{Images/launch_traffic_LEO.png}\caption{Payload launch traffic into LEO (1957-2019). \textit{Source: \cite{ESA 2020}}}
\label{launch_traffic_LEO} 
\end{figure}

\begin{figure}
\centering
\includegraphics[width=0.9\textwidth]{Images/launch_traffic_GEO.png}\caption{Payload launch traffic into GEO (1957-2019). \textit{Source: \cite{ESA 2020}}}
\label{launch_traffic_GEO} 
\end{figure}

%Why commercial space industry is active
The reasons why the commercial space industry is more active in the last decade are various. Firstly, there is a growing demand for information services, which increases competition and reduces the costs in the market. Also, there is greater availability of capital compared to previous eras of commercial satellite growth and at the same time increasing affordability of access to space launch. \cite{Hallex} Apropos this affordable launch offers, in 2019 \textit{SpaceX} announced its rideshare program offering launch capacity for smallsats charging four to eight times cheaper than most options currently available. Since the launch is one of the most expensive aspects of a space mission, such an offer makes a big difference and creates more affordable circumstances for placing into orbit small satellites. \cite{Erwin}

\bigskip
\subsection{Proliferated constellations}
\bigskip

Another important fact is that approximately from the beginning of the millennium and for the first time in history, there is a percentage of launched objects, which are part of constellations. Moreover, since 2017 the number of objects taking part in constellations is increasing as it can be seen in Figure \ref{constellation_count_LEO}. Nevertheless, the mass of these objects has a steady course throughout this aforementioned period. (Figure \ref{constellation_mass_LEO})

\begin{figure}
\centering
\includegraphics[width=0.9\textwidth]{Images/constellation_count_LEO.png}\caption{Evolution of number of objects as constellations at LEO. \textit{Source: \cite{ESA 2020}}}
\label{constellation_count_LEO} 
\end{figure}

\begin{figure}
\centering
\includegraphics[width=0.9\textwidth]{Images/constellation_mass_LEO.png}\caption{Evolution of mass as constellations at LEO. \textit{Source: \cite{ESA 2020}}}
\label{constellation_mass_LEO} 
\end{figure}

The emergence of the large constellations and the further expansion of this mission design has both positive and negative effects. An important point in favor of the proliferated constellations is that they seem propitiatory for the future. This is because it provides more resilience since the loss of a satellite will not hamstring the whole constellation. Thus, the mission can continue due to the fact that there is redundancy. The proliferated constellations have in fact some drawbacks as well. Given that the large fleets are mass-produced, the performance of each and every satellite has reduced reliability. Thus, there is a possibility of malfunction before it is brought back. 

\bigskip
\subsubsection{Focused on the field of global internet}
\bigskip
%Proliferated constellations ---In persistent global internet --- Advantages n Disadvantages
The mega-constellations or else proliferated constellations are in most cases aiming at providing persistent global internet. Their focus is to provide low latency communication, which is competitive with the terrestrial broadband one. Despite the numerous new satellites being placed into orbit and the intensification of the already existed problem of overpopulation, there are some advantages of the large constellations focused on communication. As the developing world lacks access to terrestrial broadband infrastructure and almost half percent of the global population does not have access to the Internet, such mega-constellations appear to be helpful towards solving this problem. Another positive effect is that the laying of the costly fiber-optic cable in the developing world will be skipped. Last but not least, the high-latitude populations, such as Alaska, parts of Canada, Scandinavia, Russia, will also have the possibility of connectivity and communication with the rest of the world. \cite{Hallex}
%"Their aim is also to provide high bandwidth". + Launch of mega-constellations for satellite broadband % =ευρυζωνικότητα > Internet high-speed access.

\bigskip
\subsubsection{Focused on the field of EO}
\bigskip
%Proliferated constellations ---In EO --- Advantages n Disadvantages
Similarly to the field of Communications, large constellations focused on Earth Observation have been realized. Since 2014, as can be seen in Figure \ref{launch_traffic_type_LEO}, the number of the launched payload dedicated to imaging and technology has been increased rapidly. Correspondingly to the field of communication, there are several advantages of the constellations focused on EO. Many EO satellites can result in a rapider revisit time over points on Earth. This enables shorter intervals between image captures and thus the changes that appear on the ground can be easier characterized if they were driven by human activity or unanticipated events. Furthermore, the likelihood of getting a cloud-free image, or an image with fewer shadows is crucial for further analysis. They can also provide more rapid response to global events and enable imaging at times of the day previously unseen by satellites. As a result, the countries can meet carbon neutrality targets, track and quantify methane emissions and the climate change objectives are pursued. \cite{LE_Esteve} % Monitor deforestation \& human activity in general
%"This has increased the access to Earth Observation capabilities useful for national security applications."
% More notes about Planet in docx. Source: "https://www.planet.com/pulse/12x-rapid-revisit-announcement/" -Maybe not necessary to mention it.
% The ability to image all of Earth's landmass every day was reached by Planet in February 2017. 149 "Dove" satellites are (is still the same number?) orbiting Earth. [Is it an advertising?!]
%The last launching of those satellites, which were 88, is also known as "Flock 3p". Source: https://www.planet.com/pulse/planet-launches-satellite-constellation-to-image-the-whole-planet-daily/

\begin{figure}
\centering
\includegraphics[width=0.9\textwidth]{Images/launch_traffic_type_LEO.png}\caption{Evolution of the launch traffic at LEO per mission type. \textit{Source: \cite{ESA 2020}}}
\label{launch_traffic_type_LEO} 
\end{figure} 

\bigskip
\subsubsection{The challenge of data handling}
\bigskip
% Many satellites == Hundreds of data / petabytes.
Besides the advantages that the services of the mega-constellations bring, there are challenges that need to be faced with regard to data handling and management. With so many satellites orbiting Earth, the existing database of EO services collectively is expected to exceed hundreds of petabytes globally. As the company \textit{Planet} forecasts, upon completion of the next \textit{Dove} constellation, there will be received up to 40 terabytes of data every day. In the same manner, the \textit{Sentinel} constellation is estimated to produce more than 20 terabytes of data per day. \cite{LE_Esteve}

\bigskip
\subsubsection{Examples of large constellations}
\bigskip
Some examples of the larger communication and EO constellations that are operated today or planned to be launched are the following. In the field of communication, the biggest constellations are operated by \textit{SpaceX} (\textit{Starlink}), \textit{Amazon} (\textit{Project Kuiper}), \textit{OneWeb}, and \textit{Telesat}. More information can be found in the Table \ref{table:internet}. Regarding the proliferated constellations of EO, the most renowned are operated by \textit{Planet, Spire Global}, and others, which can be found along with additional information in the Table \ref{table:EO}.

%By the end of 2017, Planet operated a constellation of 140 Dove imagery CubeSats, 5 RapidEye medium-resolution --> there are not operational since ~ 2015, and 13 higher resolution SkySat satellites that can image Earth’s entire landmass daily.

%In July 2018, Spire operated 61 of its Lemur satellites (out of a planned 125) that track the Automatic Identification System (AIS) beacons of ships that collect weather data by monitoring the radio occupation of GPS signals." (Table of EO constellations - in docx) --> I don't know the swath width..and what is the resolution of such a mission?
I saw here: https://directory.eoportal.org/web/eoportal/satellite-missions/l/lemur --> "Along-track resolution = 200km" (what is it?) \cite{Hallex}

\bigskip
\begin{center}
\captionof{table}{List of the most prominent proliferated constellations in the field of internet provision. \textit{Source: \cite{Hallex}, \cite{Amazon}, \cite{Oneweb_bankruptcy}, \cite{Kramer 2002}, \cite{Hongyun}, \cite{Viasat}, \cite{Kepler}}}
\vspace{3mm}
\begin{adjustbox}{width=1\textwidth}
\begin{tabular}{||c| c |c |c |c |c||}
\hline
\textbf{Constellation} & \textbf{Manufacturer} & \textbf{Proposed Satellites} & \textbf{Satellite Design Life (Years)} & \textbf{Altitude (km)} & \textbf{Sat. mass (kg)}\\
\hline \hline 
Starlink $^*$ & SpaceX & 12,000 & 5-7 & 550 & 250 \\
Project Kuiper & Amazon & 3,236 %\cite{Amazon}
& n/a & 590-610 & n/a\\
OneWeb $^*$ & OneWeb & 650 %\cite{Oneweb_bankruptcy}
& 5 %\cite{Kramer 2002}
& 1,200 & 147\\
Telesat & Airbus, SSTL, SS/ Loral & 292-512 & 10 & 1,000 \& 1,250 & 150\\
Hongyun & CASIC & >150 %\cite{Hongyun}
& 1 & 1,100 & 247\\
GEN-1 $^*$ & Kepler Communications & 140 & 3-5 & 520-600 & 12-15 \\ %3U to 6U cubesats (1U is approximately 1.33kg + SSO (Sun-Synchronous Orbit). In reality the GEN-1 will be consisted of up to 15 satellites. There will be also the GEN-2 with up to 50 satellites. In total the whole Kepler constellation will be consisted of up to 140 satellites.
LeoSat & LeoSat & 78-108 & 10 & 1,400 & 1000\\
Boeing & Boeing Manufacturer & 60 %\cite{Viasat} 
& 10-15 & 1,200 & >100\\

% The company SES and the constellation O3b is operating in Geostationary and medium orbits, but there are discussions of placing satellites in LEO as well.
\hline
\end{tabular}
\label{table:internet}
\end{adjustbox}
\end{center}
\footnotesize{$^*$ {\scriptsize Already partially available as of June 2020}}
\bigskip

\normalsize

\bigskip
\begin{center}
\captionof{table}{List of the most prominent commercial proliferated constellations in the field of EO. \textit{Source: \cite{Hallex}, \cite{Satellogic}, \cite{Newspace}, \cite{Kramer 2002}}}
\vspace{3mm}
\begin{adjustbox}{width=0.9\textwidth}
\begin{tabular}{||c| c |c |c |c||}
\hline
\textbf{Constellation} & \textbf{Manufacturer} & \textbf{Proposed Satellites} & \textbf{In orbit} (as of October 2020) & \textbf{Resolution (m)}\\
\hline \hline
Flock/ Dove & Planet & 150 & 150 & 3-5\\%Imaging the entire Earth daily
Lemur/ Minas & Spire Global, Inc. & 150 & 127 & n/a\\
Whitney & Capella Space & 40 & 1 & 1-30 (SAR)\\
Skysats (Terra Bella/ Skybox) & Planet & 21 & 21 & 0.5\\
Aleph-1 & Satellogic & 300 & 7 & 1\\ %Argentina + Imaging the entire Earth every 2h
BlackSky & Spaceflight Industries & 60 & 6 & 1\\%It has 3 year orbital lifetime
SuperView (GaoJing) & SpaceWill (SpaceView) & 16 & 5 & 0.5\\%China
Landmapper & AstroDigital & 25 & 5 & 2.5 \& 22\\
KOMPSAT & Earth-i & 4 & 4 & 0.4-1\\
GRUS & Axelspace & 50 & 4 & 2.5\\
UrtheDaily/ OptiSar & UrtheCast & 24 & 4 & 5\\
DMC3/ TripleSat & Earth-i & 3 & 3 & 1\\
Vivid-i & Earth-i & 15 & 1 & 1\\
CE-SAT & Canon & 100 & 1 & 0.9\\%Japan
APEX, SUNSTORM & Reaktor Space & 36 & 1 & 20\\%Finland
REC & SatRevolution & 1024 & 1 & 0.5\\
HOPSAT-TD & Hera Systems & 50 & 1 & 1\\
NorthStar & NorthStar & 40 & 0 & n/a\\%Canada
WorldView Legion & Maxar & 6 & 0 & 0.3\\%Maxar: SSL, Digital Globe, MDA, Radiant
HySpec & HySpecIQ & 2 & 0 & 1\\ %EO \& Hyperspectral


%% Not exaactly Earth Observation (EO):
%HawkEye 360 --> RF Spectrum Monitoring (Mapping). It probably does geolocation: locates the geographical (latitudinal and longitudinal) location of an Internet-connected device.
%SpaceBee satellites and others - Swarm Technology (company) - Proposed satellites: 600 - In orbit: 9 (as of 10th Sept) --> It is an IOT company.

%% Others - but not much info and not large constellations:
%Triton-1,2,X (Triton 1,2 were launched in 2013 and duration was for some months. Triton-X hasn't been launched.) - LuxSpace (OHB subsidiary company)
% ESAIL (microsatellite. It was launched in Sep20. It is just one satellite in the field of ship tracking and maritime situational awareness.) - LuxSpace (OHB subsidiary company) --> https://space.skyrocket.de/doc_sdat/esail.htm

%% These are not commercial:
%Pleiades $^*$ & CNES & 2 & 0.5-2\\ (VERY OLD - 2011, 2012)
%RADARSAT (RCM) & Canadian Space Agency & 3 & SAR (1x3)m\\ (VERY OLD - 2007)
%A-train & NASA & 5 & (Goal: Hurricanes, Climate Change)\\

%% Retired constellations:
%RapidEye & Planet & 5 & 5\\
%SPOT & SPOT Image & 2 (They were 7 in total) & 1.5-6 --> Its still operating, but its VERY OLD
\hline
\end{tabular}
\label{table:EO}
\end{adjustbox}
\end{center}
%\hspace{1.5cm} \footnotesize{$^*$ {\scriptsize Already partially available as of October 2020}}
\bigskip

\normalsize

In the following chapters, a table containing EO satellites of the commercial and non-commercial sector will be available.

Lastly, it is worth mentioning that the competition and commercialization that exists in the field of Telecommunications and Earth Observation leads undeniably to the prosperity of the economy and it is also essential for preventing the creation of quasi-monopolies. Considering as a hypothetical scenario the case of \textit{Starlink}, \textit{SpaceX} while having no competitors and being successful with launching all the 12,000 proposed satellites; it will result in depleting most of the available resources in space. Therefore, for preventing such cases, the goal of the proposed application of this thesis is to recognize and encourage a virtual competition between different entities. This will help towards finding the best use of the available hardware in orbit instead of launching entire constellations and exhausting the remaining spatial resources.

% There will be competition in the lab, not in space!

All in all, NewSpace provides cheaper access to space and at the same time affordable data and services. More and more small satellites are launched in LEO in constellations, which are funded by the commercial sector. Even though this trend has several benefits, spatial and space awareness should be raised in order to promote the use of proliferated constellations, but also avoiding a proliferated LEO. \cite{pLEO} In the same manner, by diminishing the launch rate and maintaining the growth of the services that are offered from space can lead to the reduction of space traffic. % to an extent.

%NewSpace (or Space 2.0): a movement and philosophy encompassing a globally emerging private spaceflight industry. Independent from governents and major contractors. (from Wikipedia)

%Question about title: Why the thesis can reduce space traffic? Ans: It gives the incentives. By sharing resources we have a chance to reduce space traffic.
%One question: "I want a certain EO task" --> What resolution can I have without exceeding orbit capacity?"
\chapter{Theoretical Framework}
\lhead{Chapter 2 \emph{Theoretical Framework}}
\label{chap:2}
%\autoref{chap:2}

\section{The problem of overpopulation}
\bigskip

Space is a non-infinite environment and an inherently %=έμφυτα, εγγενώς
international natural resource. At the same time, there is no ownership in space, as it is stated by UNOOSA in the Outer Space Treaty. However, it seems that nowadays it can be linked to the phenomenon of Tragedy of Commons written by the British economist Lloyd. Based on this concept, the shared-resource system is spoiled by individual users who act based on their own benefit and against the common good. This notion describes the current situation of the space industry, which comes as an international problem.

\bigskip
The source and sink mechanisms, which explain how the outer space has become congested, were described by Somma et al. [2018]. This aforementioned work \cite{Somma 2019} presents the equations of the injection and removal of an object from any altitude range from 200 to 2000 km. As it can be seen in Figure \ref{mechanism}, new objects are created by launches, explosions and collisions. Conversely, an object can be removed due to natural drag, post-mission disposal, and active debris removal.

\begin{figure}
\centering
\includegraphics[width=0.9\textwidth]{Images/mechanism.png}\caption{Source and sink mechanisms depicting an object's injection and removal from a specific altitude. \textit{Source: \cite{Somma 2019}}}
\label{mechanism} 
\end{figure}

The total population of objects $\vec{N_T}$ at every discrete time $t_k$ in a specific altitude shell $h$ can be found as:
\begin{equation} \label{total_population}
\vec{N}_T(t_k, h) = \vec{N}_{AP}(t_k, h) + \vec{N}_{IP}(t_k, h) + \vec{N}_{RB}(t_k, h) + \vec{N}_{MR}(t_k, h) + \vec{N}_{CO}(t_k, h) + \vec{N}_{EX}(t_k, h),
\end{equation}

where the subscripts $AP$, $IP$, $RB$, $MR$, $CO$, and $EX$ refer to the objects of the following categories respectively: active payloads, inactive payloads, rocket bodies, mission related objects, collision and explosion. \cite{Somma 2019}

\pagebreak
In a similar manner, the derivative of the total population in each altitude shell can be found as:
\begin{equation}\label{derivative}
\dot{\vec{N}}_T(t_k, h) = \dot{\vec{N}}_{AP}(t_k, h) + \dot{\vec{N}}_{IP}(t_k, h) + \dot{\vec{N}}_{RB}(t_k, h) + \dot{\vec{N}}_{MR}(t_k, h) + \dot{\vec{N}}_{CO}(t_k, h) + \dot{\vec{N}}_{EX}(t_k, h).
\end{equation}

Hence, the total population of the objects in any discrete time $t_k$ can be predicted using:
\begin{equation}\label{future}
\vec{N}(t_{k+1}) = \vec{N}(t_k) + \dot{\vec{N}}(t_k, \vec{N}(t_k))\Delta t.
\end{equation}

It can be concluded that if the derivative of the total population (\ref{derivative}) is positive then the objects being added outnumber the objects being removed, which leads eventually to congestion in those altitude regions of the outer space.

\bigskip
%There is a big problem especially in LEO
The above mentioned problem of overpopulation is mainly encountered in LEO. To be more exact, approximately 75\% of all cataloged objects are located in LEO, which results in several negative effects. Not only the risk of collision becomes bigger, but also the safety of human spaceflight is at risk, since ISS operated and performs in the LEO altitudes (\textasciitilde 350-400km). \cite{Kramer 2002} The effect of the growing congestion in the outer space was also noted by the CEO of a launch company mentioning that it becomes harder to find a path for rockets to launch new satellites. \cite{crowded}


\bigskip
\subsection{The reasons behind overpopulation}
\bigskip
%Reason why it happens/ Causes
The reasons behind the problem of overpopulation are manifold. Firstly, one root cause is the incremental number of launched objects, which is also highly connected to the free market, the commercial district and the free competition between the private corporations. %(no trading restrictions)
One argument to launch more satellites by the \textit{Planet} company is that the commercial market has opened up since the daily revisits have been increased with the newly launched satellites and the offered resolution is higher. As a direct result, there is more demand as the costumers show greater interest, which is cited as a good reason to launch more objects. \cite{CNBC}

Even though the risk of collision is one of the side effects of overpopulation, in the event of an actual impact, the particles being created will definitely deteriorate the situation. This is in fact the vicious circle predicted by NASA's scientist Kessler. (Chapter \ref{chap:1.1}) Thus, as a second reason of the problem of overpopulation, the accidental break-ups can be mentioned. The first accidental collision, which took place in 2009 at LEO, is undoubtedly a benchmark in space history. The number of debris left behind from this collision between the US \textit{Iridium-33} and the Russian \textit{Kosmos-2252} was more than 2,300 pieces with average size of 12.5cm and average weight 1.1kg. \cite{Kelso 2009} This impact is an indication that a collisional cascading might happen soon, if action is not taken.
%Between the US commercial non-operational Iridium-33 and the Russian military Kosmos-2251.

Another reason behind the overpopulation problem is the satellite interceptions by surface-launched missiles. The anti-satellite tests are weapons, which not only aim at destroying satellite, but also their action creates uncountable space debris. As an example, in 2007, the intentional destruction of the chinese \textit{FengYun-1C} satellite doubled the amount of debris at an altitude of about 800 km, leading to a 30\% increase in the total population of debris at that time. \cite{Anti-satellite} Last but not least, the fact that there is lack of space traffic regulations, and mitigation guidelines, which are legally binding under international law, such problem is difficult to be handled.

\bigskip
\subsection{Negative effects \& temporary ways of collision avoidance}
\bigskip

Some of the negative effects of this crucial issue has already been emerged as they are discussed below, whereas some others are looming. The number of launched objects and the number of space debris is proportional to the possibility of impact between two objects/ particles. In other words, in the case of overpopulation, the risk of collision is higher. More specifically, after the first accidental aforementioned collision in 2009, the remote sensing satellites \textit{ERS-2} and \textit{Envisat} due to their proximity to the region of impact, have increased their risk of a secondary catastrophic collision by almost a factor 2. \cite{Klinkrad 2009}

Another possible aftermath of the problem is linked to the deterioration of socio-economic conditions. Important space services could be lost, such as weather forecasting, emergency management, as well as climate monitoring. Additionally, space-based communications and internet could be affected, resulting in a required disruption of the communication and transactions globally. \cite{Undseth}

\bigskip

A potential risk of collision has led to find another ways of preventing clash between satellites. These are the creation of shields for protecting satellites from space debris, the need of surveillance and tracking of the orbiting objects, and the replacement of missions altogether to be equipped with the latest technology and to be able to maneuver if needed. Even though these measures had proved to be helpful, their operation do not solve the root of the problem.


\bigskip
\section{Mitigation principles of overpopulation}
\label{chap:mitigation}
\bigskip

\bigskip
\subsection{Taking action}
\bigskip

\subsubsection{Debris monitoring}
There are multiple ways to deal with the problem of overpopulation. %confronting a problem
To begin with, the methods, which help towards the avoidance of collisions between the already launched satellites are the following. One approach is to track debris by using ground-based systems. Objects larger than 10cm in LEO and larger than 1m in GEO can be tracked. \cite{Kramer 2002} However, according to a recent study by the University of Warwick and the Defence Science and Technology Laboratory in the UK was found that more than 75\% of the objects that can be observed and that are bigger than 1m in GEO, could not be classified. It is absolutely urgent to continue observing the space debris in the GEO region in order to understand deeper what is the current situation before launching more objects. \cite{Blake} %Article in which I found it: https://newatlas.com/space/orbital-space-debris-threat-active-satellites/
 Smaller objects than the aforementioned sizes, not to mention objects in the mm scale, are difficult to be detected due to the sensitivity limits of the instruments, such as radars and telescopes.

\bigskip
\subsubsection{Collision avoidance}
Another way of dealing with the problem, which requires more effort, time and bears extra costs, is the method of collision avoidance messages and subsequently the action of maneuvering a satellite. In case there is an estimation of a possible collision with at least one of the involved objects being operational, maneuvering is the instinctive reaction of the satellite operators.

\bigskip
\subsubsection{Active space debris removal}
In a similar manner, a method, which will have a great impact on reducing the risk of collision especially with the big, non-operational satellites that are still orbiting Earth, such as \textit{Envisat}, is the active space debris removal. There are many other ways of space debris removal, such as the use of robotic arm, clamping mechanism, net, magnets, foam or even by using a collision with small relative velocity.\cite{active}, \cite{takeichi} Some of the projects working in this direction is the \textit{e.Deorbit} by ESA, the \textit{RemoveDEBRIS} by EU and University of Surrey, as well as the EPFL's \textit{CleanSpace One}.

\bigskip
\subsubsection{Deorbiting methods}
Several ways of deorbiting can also help into mitigating overpopulation. The uncontrolled reentry is one of them, which occurs when the satellite is approximately at 600km and it is also referred as natural drag. This way of uncontrolled reentry is proportional to the ballistic coefficient, which is the effective drag area divided by the mass. Thus, in case of having a small satellite, the decay is longer as compared to the decay of a larger one. Although this method is a popular choice of dealing with the overpopulation, it still poses a threat, since the satellite passes unguided through lower altitudes. On the other hand, there is the controlled reentry, which is a deorbiting option when the satellite is located approximately at 800 to 1,000km. In this method, the use of thrusters or deployment of drag devices is one of the alternatives. Furthermore, when the satellite is in higher altitudes than of 1,000km, a considerate thought is to maneuver the satellite to one of the disposal regions above the 2,000km, in which the objects will not interfere with the future space operations. \cite{NASA} Another way of deorbiting, which was developed recently in the Chukyo University is by using drag force intensifier applying charged membrane. \cite{muranaka}


\bigskip
%A method, which prevents the further acting of deorbiting after the end of life of a LEO satellite is its placement to a MELVOs orbit (Moderately Elliptical Very Low Orbits) instead of a LEO one.
\subsubsection{The benefits of MELVO orbit}
A convenient method, which has been prepared and programmed for the deorbit of the satellite without further action after the end of life due to the orbit selection, is the following. Instead of placing a future LEO satellite in a LEO orbit, it is placed in a MELVOs orbit (Moderately Elliptical Very Low Orbits). The characteristics of a MELVO orbit is that the perigee and apogee are in a lower altitudes of 300 and 500km respectively and the eccentricity is between 0.015 and 0.030. Some of the advantages of this method is that when the orbit maintenance stops, the debris population can decay in this altitude and reenter the atmosphere in a short time, due the elliptical shape of the orbit. As a reference, from a circular orbit of 300km, it reenters the atmosphere in about 23 days at solar maximum and 70 days at solar minimum. Also, if there is a failure in the orbit maintenance burns, the orbit will turn into a circular one without loss of delta V. Another positive aspect of this idea is that since the perigee has low altitude, there is a better resolution on Earth and also reduced orbital debris issues, both in terms of collision probability and contribution to the long term debris problem. \cite{Kramer 2002} On the other hand, the disadvantages of this concept are that delta V is required, the coverage is reduced compared to a LEO orbit and that it has a reduced design life. Nevertheless, it has potentially lower cost per year.



\bigskip
\subsection{Raising consciousness}
\bigskip

Several important steps have been made towards raising the consciousness about the overpopulation problem, as well as the enactment of rules regarding the capacity and orbit allocation. The most significant milestones that showed an international cooperation at a technical level are the following.

\bigskip
\subsubsection{First space debris mitigation guidelines}
\bigskip

In 1993 the \textit{Inter-Agency Space Debris Coordination Committee} (IADC) was founded. Its goal is to coordinate the efforts among the various space agencies that are involved and structure a plan for dealing with the debris orbiting the Earth. Some years later, in 2002 IADC published for the first time a report about Space Debris Mitigation Guidelines. \cite{UNOOSA} %The space debris mitigation was performed in 2002 by IADC and presented at UNCOPUOUS in 2003.
These guidelines were presented at the \textit{United Nations Committee on the Peaceful Uses of Outer Space} (UNCOPUOUS), which is a committee established by \textit{United Nations Office for Outer Space Affairs} (UNOOSA). \cite{IADC 2007} In 2003 another group focused on similar goals was created – the \textit{Orbital Debris Co-ordination Working Group} (ODCWG). It was established by unanimous agreement of \textit{International Organization for Standardization} (ISO). \cite{Klinkrad 2006}
%The two main groups that work towards these goal are the UNCOPUOUS and the ODCWG.

The first five mitigation guidelines, which were presented at the UNCOPUOUS in 2003, are related to the following concepts: prevention of the release of a mission related object, implementation of collision avoidance measures, prevention of explosions by releasing stored energy, disposal, passivation and limitation of the on-ground risk due to re-entry. Some other guidelines for the current debris mitigation are: the avoidance of intentional generation of debris, such as anti-satellite tests, as well as the 25-year deorbit rule for the LEO missions. As it was mentioned in the ESA's latest Space Debris Environment Report, less than 60\% of the LEO satellites comply to these guidelines. \cite{ESA 2020} %(from http://www.esa.int/Safety_Security/Space_Debris/The_cost_of_space_debris)

Even though there are national and international mitigation measures, the compliance is insufficient to stabilize the orbital environment. For this reason, it was investigated that by reducing the residual lifetime from 25 to 5 years we could prevent the increase of inactive population and thus the prevention of collision fragments. The commercial operators should commit to design objects with lower residual life. Thus, the satellite deorbit reliability will be maximized. \cite{Somma 2019}

The aforementioned Space Debris Mitigation Guidelines are applicable not only to mission planning and designing new spacecraft, but also to the existing orbiting objects if possible. The implementation of those guidelines is highly recommended, however it is a voluntarily action due to the fact that they are not legally binding under international law. \cite{UNOOSA}

% NO GUIDELINES SINCE 1967:
%The 1967 Outer Space Treaty, which remains the primary international document regulating activity in outer space, was agreed to at a time when only two governments were going to space. Now that more countries and commercial companies are also in the business of spaceflight, regulators are faced with a Catch-22: They don't want to create a lawless environment, but they're reticent to impose new rules for fear that other countries may become more dominant in space.
%Recent attempts to update rules on the international stage have been "incredibly inspiring, but also incredibly depressing," Beck said. Because even though countries were willing to come to the table, nothing has actually been agreed upon since the 1970s. Source: https://edition.cnn.com/2020/10/07/business/rocket-lab-debris-launch-traffic-scn/index.html

\bigskip
\subsubsection{Comprehensive tool for space debris mitigation guidelines}
\bigskip
One of the difficulties that a mission faces in the designing phase is the compliance to the space debris mitigation guidelines. The reason that lies behind is the existence of various regulations from different entities, such as UN, EU, ECSS, ISO and others. In order to be tackled, ESA has developed a software the so-called Debris Mitigation Framework (DMF), in which all the necessary information and tools regarding the requirements, standards and guidelines has been collected. So, DMF is the link between all the needed information about a mission, which has been specified by the user. \cite{Braun}

\bigskip
\subsubsection{Space sustainability rating}
\bigskip
%SPACE SUSTAINABILITY RATING - HOW WELL AN INDIVIDUAL SATELLITE FOLLOWS THE GUIDELINES.
The fact that the existing international guidelines are not enforceable and the derived standards are not always followed, has led the \textit{World Economic Forum} (WEF) to select a consortium of companies, universities and agencies in order to create a system for rating the sustainability of space systems. \cite{Space sustainability}  %Source: https://www.weforum.org/projects/space-sustainability-rating
In particular, this project is consisted of an international and multidisciplinary team and the main collaborating entities are the \textit{European Space Agency} (ESA), \textit{Massachusetts Institute of Technology} (MIT), \textit{University of Texas at Austin} and \textit{Bryce Space and Technology}. The so called Space Sustainability Rating (SSR) concept has the goal of encouraging a responsible behavior in space through increasing the transparency of organization's debris mitigation efforts.
%Francesca Letzia and Stijn Lemmens (from ESA) are working for that. Another idea would be also to have the concept of added-value incorportated into that rating. But still it seems very difficult - so it is not part of the parameters.
%The other side of the insurance companies say: If one constellation has a good rating, then the insurer (insurance company) will make a discount. But at the same time this will probably affect the requests for insurance claims on satellites (since the rating is good and thus the risk as well of collision is small - the satellite companies might not want to be insured). *These are just guesses...!*

In order to ensure the long-term sustainability of space, a measurement for every individual satellite or constellation will be carried out, which will show the degree of commitment to the guidelines. 
%The SSR will give a score related to debris mitigation and alignment with international guidelines. (I said it with other words.)
The calculation of this score will take into account both quantitative and qualitative parameters. Some determinant parameters are each satellite's design and physical parameters, the economic viability of the satellite's operator (relevant example of \textit{OneWeb} \cite{Oneweb_bankruptcy}, \cite{Cadman}), % In case, a company declares bankruptcy, as OneWeb (UK) did, many already launched satellites are doomed to stay in LEO occupying space in that area and increasing the risk of collision when they will be become non-operational. (There are now in orbit around 70 OneWeb satellites). Chinese investors are interested into buying the company. The United States and other Western nations are not keen on the idea of seeing OneWeb slip into "adversarial hands.") British government and Indian mobile network operator Bharti Global placed the winning bid to acquire OneWeb (1 billion$).
how trackable the satellite is, estimations concerning potential collisions and the satellite's disposal at the end of its life. 

Currently there are so many different definitions of space sustainability and this term needs to be shaped and generally accepted. Nevertheless, this level of sustainability of a mission will be publicly available by placing emphasis on the debris mitigation approach and by giving a positive incentive to satellite operators to increase their responsible behavior and to improve their ratings. An initial beta version of this rating system is going to be released at the end of 2020/ beginning of 2021. \cite{Space sustainability}
%Source: https://www.weforum.org/projects/space-sustainability-rating

\bigskip
\subsubsection{Concept of orbital capacity}
\bigskip

The concept of orbital capacity was formulated in order to estimate the global evolution of the space environment. Every orbital plane has a certain capacity, which should be taken into account when a future satellite constellation is going to be launched. Towards this goal, the calculation of the Space Debris Index is essential. It is based on the collision risk of operational satellites and the collision effect, as well as the explosion probability and effect. This index can be applied to both single objects and to the whole environment. As a direct consequence, when there is more capacity, then the reliability, which is connected to the Post Mission Disposal success rate, is low. The concept of orbital capacity will be used to sharpen the current space debris implementation guidelines. \cite{Letizia 2019}

In this way, along with the concept of "Space Sustainability Rating" (SSR), the concept of "Space Traffic Footprint" will be created. It will include data on the size and location of an object, how crowded the orbital area is, what already exists there and if a satellite is maneuverable and trackable. \cite{Space sustainability}
%Source: https://phys.org/news/2019-05-space-sustainability-aims-amount-debris.html

\bigskip
Some other methods that can help into mitigating the problem of overpopulation are the following. The concept of added value is based on the estimation of the characteristics of a satellite or a constellation before they are launched. In this way, the launching of the missions can be prioritized based on the added value that give to the society. Another idea of dealing with the overpopulation is the concept of the cooperation between the entities towards sharing common resources. More details about those concepts can be found in the following section (Section \ref{section}).


\bigskip
\section{Mitigation concepts encouraged by the current work}
\label{section}
\bigskip

\subsection{Estimation of added value}
\bigskip
%ESTIMATION OF ADDED VALUE
As it was mentioned in Chapter \ref{chap:scope}, the goal of this thesis is to give the incentives for the formation of rules which will be based on the added value that a new satellite offers. As an example towards this direction is the report of the independent consultancy, London Economics (LE), which is working in the space sector, regarding the value of the EO capabilities offered by satellites to the UK government. \cite{Value UK} In this case the value is determined based on multiple parameters such as the operational cost saving, the exceptional cost avoidance, better policy decisions and the benefits that the government, the economy and the society gain from the offered EO satellite capabilities. This study can be a great approach and can serve as an example to other nations and institutes towards prioritizing the missions based on their potential value to the society.

In the current work, the developed application offers the opportunity to a user to find out how many satellites have the same characteristics, such as same sensors, similar revisit time, common field of utilization, such as Environmental monitoring - etc.

\bigskip
\subsection{Cooperation between entities towards sharing resources}
\bigskip

A different way of mitigating the problem of overpopulation, which is also connected to the aforementioned idea of added-value, is the concept of the cooperation between entities towards sharing common resources. As it has also been mentioned in the GEO (\textit{Group on Earth Observations}) summit in Tokyo, 2004, when the GEOSS (\textit{Global Earth Observation System of Systems}) was created, in order to have a societal benefit of Earth Observation, data sharing is necessary. \cite{Kramer 2002} %Krmaer p.39
The Argentinian and the Italian space agency, CONAE and ASI respectively, have showed interest in this direction. They have the vision of integrating the two national systems by using their EO constellations, \textit{SAOCOM} of Argentina and \textit{Cosmos-SkyMed} of Italy in order to generate information for the sector of Emergency Management on environmental issues. \cite{Cosmos}  %SAR sensors

Another example is the group \textit{BRICS} consisted of the states Brazil, Russia, India, China and South Africa, which will join their forces towards a mission regarding natural resources and disaster management. They will use already existing/ orbiting Sun-Synchornous Remote Sensing satellites. \cite{BRICS} An initiative was also taken from UNOOSA under the name of \hspace{1mm} “Access to Space for All”, which has the goal of connecting and creating necessary conditions for cooperation between the established space actors in order to achieve the SDGs (Sustainable Development Goals). \cite{UNOOSA}
%Other links related to the symposium in 2020: https://www.unoosa.org/documents/pdf/copuos/stsc/2020/acces2space4allprogE.pdf + https://www.unoosa.org/oosa/en/ourwork/copuos/stsc/2020/unoosa-symposium.html

In the same manner, the National Reconnaissance Office (NRO) of USA searches for commercial companies in order to support the US government missions. The procedure that is followed is the inspection of each company's current and planned capabilities, as they want to have more than one collaborator/ provider. This is noteworthy, since not only this request gives the incentives for collaboration among companies, but also the creation of a state of monopoly in space is certainly undesirable. \cite{NRO}

A more recent example of entities which have cooperated with each other towards a common goal is between ESA and NASA in order to measure Antarctic sea-ice. This joint-campaign is known as Cryo2Ice with the following satellites participating, CryoSat-2 (ESA) and ICESat-2 (NASA). This cooperation will involve orbit synchronization, since the number of coincident observations observed by the two satellites needs to be increased. The instruments that the two satellites carry on-board are different. More specifically, NASA's IceSat-2 has laser, which measures the distance to the Earth's surface and thus the height of objects. On the other hand, ESA's Cryosat-2 uses radar, which is penetrating more deeply to the snow than laser before bouncing back. The idea for this tandem mission observing the polar regions came from the fact that the advantages of both sensors will result in more reliable outcomes. \cite{Cryo2ice_news}, \cite{Cryo2ice}
%ESA (CryoSat-2, radar ~720km) and NASA (ICESat-2, laser  ~500km). Cryosat-2 changed orbit by raising the altitude in approximately one kilometer. "Icesat is quite a bit below us so we can't go down to meet them, but by going up we find this incredible resonant orbit in which for every 19 orbits for us and 20 orbits for them - we will meet at the poles within a certain time lag. Basically, every 1.5 days, we meet over the poles within a few hours of each other and that means we can observe the same ice almost simultaneously."

Additionally, another cooperation that has been actualized is between ESA's Copernicus Sentinel-5P and GHGSat’s Claire. The goal is to map a range of atmospheric gases around the globe every 24 hours. Collectively the different sensors of those two satellites can identify significant methane leaks, which can lead to better monitoring of the climate change. \cite{cooperation} %The same will happen with the next satellite "Iris".

% Cooperation between companies: Planet and Copernicus encourage people to use jointly their data (there is also the challenge “See change, change the world”) --> I don't think that it has the character of mitigating the problem of overpopulation etc. Or that the cooperate for this reason.

\bigskip
\section{Orbit and mission type in the current work}
\bigskip

As it was mentioned in Chapter \ref{chap:scope}, the main goal of this master thesis is the development of an application, which is focused on Low Earth Orbits in the field of Earth Observation. For this reason, the section below is comprised of the main attributes and important parameters of those two sectors.

\bigskip
\subsection{Low Earth Orbits}
\bigskip

\subsubsection{General characteristics}
\bigskip
Low Earth Orbits (LEO) are placed in the closest region above Earth's surface. Their altitude range from 200 to 2000 km, which leads to a limited swath area on the surface of the globe. They are characterized by short orbital periods between 88-127 min and thus many revolutions per day. Their orbital velocities are high, around 7-8 km/s with regards to an observer on Earth and the inclinations that can have are all possible. \cite{Campbell}

The LEO is also a favorable spot for placing satellites for various reasons. It's proximity from Earth makes it possible to launch even CubeSats, since their fuel capacity is enough to be able to maintain their position and/or to de-orbit at the end of their mission. Specifically in the recent years more and more small satellites and/or CubeSats are placed in LEO and in large constellations. (Fig. \ref{launch_traffic_LEO}) The need of proliferated constellations is also linked to the low altitude of the LEO region. Namely, for larger coverage, a satellite should be either placed in a higher altitude, or the mission should engage more satellites in order to compensate the smaller field of view that one satellite offers.

The main areas that LEO is used are for remote sensing, earth observation, global positioning, navigation, human spaceflight, climate, weather forecast, military, and others. 

\bigskip
\subsubsection{LEO subcategories}
\bigskip

\begin{itemize}
\item Classification based on eccentricity
	\begin{itemize}
	\item Circular LEO: \\
	It is the most common and natural orbit for Earth Observation. The typical altitude range from 200 to 900 km with orbital periods of about 90-105 min.
	\end{itemize}
\item Classification based on inclination
	\begin{itemize}
	\item Equatorial or tropical orbit: \\
	The inclination of this orbit is zero or close to zero degrees. It provides close coverage of the tropical regions.
	\item Polar or near-polar orbit: \\
	
	\item Sun-synchronous orbit
	\end{itemize}
\end{itemize}



	Polar orbit - Near-polar orbit (inclination of 90deg or near 90deg -usually when its >60deg-, which would pass over the Earth's poles each orbit or over higher latitudes. Another benefit is that the orbital plane is fixed and the Earth rotates continuously beneath this plane. So, since the orbital period is a integral multiple of the sidereal day, then the satellite passes over the same point on the surface at regular intervals.) Their orbit which is basically north-south with conjunction with the Earth's rotation (west-east), allows them to cover most of the Earth's surface over a certain period of time.
-- The choice of altitude for a polar orbit is determined by several factors. A lower altitude orbit results in: a) a shorter orbital period, b) poorer coverage of the surface, c) stronger signal returns, d) better spatial resolution, e) greater drag and shorter lifetime.
Many of these near-polar orbit orbits are also sun-synchronous, such as they cover each area fo the world at a constant local time of day called local sun time.
	Sun-Synchronous orbit (a special case of near-polar orbit!):
An orbit like this is possible by the fact that the Earth is not a perfect sphere. In the SSO, the daily rotation of the orbital satellite plane (with respect to the equatorial plane) is identical to the mean motion of the fictitious sun around the Earth = which is identical to the mean motion of the Earth around the sun. So, if a satellite is placed in an orbit at an altitude producing a period that was an integral multiple of the sidereal day, then the satellite passes over the same spot on the Earth at the same solar time each day, every day of the year. This is a great benefit especially for the remote sensing satellites, since the area of interest can be monitored throughout the year with similar illumination conditions and the changes can be easier detected.
%(((So, a SSO it is an orbit for which the plane of the satellite orbit is always the same in relation to the Sun. A sun-synchronous orbit is not fixed in space. It must move 1° per day to compensate for the Earth's revolution around the Sun. Since the Earth makes one revolution (360°) around the Sun per year, we can calculate the rate of change of the Sun's right ascension: 0.9856473°/solar day. → this is the required rate of precession of Ω))) --Cite: put some generic like Kramer... --->  "This precession ensures that the equatorial crossing times of the satellites, in terms of the local solar time, remain nearly constant throughout the year. This means that a satellite can make repeated global observations from a single set of sensors with similar illumination from pass to pass." (Source: https://ral.ucar.edu/~djohnson/satellite/coverage.html#polar)
The inclination of the orbit (for
circular orbits) and altitudes between 400 km and 1000 km, is
in the range of 97° to 99°. Therefore, they are also quasi polar orbits, which allow the whole surface of the Earth to be
covered, even passing several times a day over the same point.
Source: Keplerian_orbits
%Meseguer, J., Pérez-Grande, I., & Sanz-Andrés, A. (2012). Keplerian orbits. Spacecraft Thermal Control, 39–57. doi:10.1533/9780857096081.39
%(Publisher: Woodhead Publishing)
%
%https://www.sciencedirect.com/book/9781845699963/spacecraft-thermal-control#book-info
"At any given latitude, the position of the sun in the sky as the satellite passes overhead will be the same within the same season. This ensures consistent illumination conditions (=lighting conditions will be nearly
the same) when acquiring images in a specific season over successive years, or over a particular area over a series of days. This is an important factor for monitoring changes between images or for mosaicking adjacent images together, as they do not have to be corrected for different illumination conditions." (Source: https://www.nrcan.gc.ca/maps-tools-and-publications/satellite-imagery-and-air-photos/remote-sensing-tutorials/acknowledgements-permission-use/9391)

Their altitudes usually range from 700 to 800 km, with orbital periods of 98 to 102 minutes.

In the case of sun-synchronous orbits with an optical sensor, it is important to know which data are taken from the "ascending" pass of the orbit, which corresponds to the pass of the orbit when the satellite is moving from south to north, and which from the "descending" which corresponds to north to south movement. Then, by knowing the revisit time we can have a better view on the outcome results. (If the orbit is sun-synchronous) then the ascending pass is most likely on the shadowed side of the Earth while the descending pass is on the sunlit side. So, in the case of optical imagery we are interested on the descending passes. In the case of active sensors which provide their own illumination or passive sensors that record emitted (i.e. thermal) radiation the outcome of both passes is useful.



  Constellations: With two or more satellites a desired coverage of the Earth can be achieved easier. "The satellites may be placed within the same orbital plane to decrease the orbital period, or the satellites may be placed into different orbital planes to compensate for the movement of the observation area due to the rotation of the Earth.
% Source: Campbell BA, McCandless W (1996) Introduction to space sciences and spacecraft applications. Elsevier


\bigskip
\subsubsection{Critical regions of LEO}
\bigskip
%
%IADC (Inter-Agency Space Debris Coordination Committee) has declared protected regions \cite{IADC 2007}.

The total number of operating satellites, as of 31.3.2020 including all the launches by that date is 2,666 with 1,918 being in LEO. (It is the most crowded one) \cite{UCS} There are some regions of LEO, which have already been declared as critical. More specifically, the three critical regions in LEO are \cite{Kramer 2002}:
\begin{itemize}
\item The first critical region is at the altitude 750-850km, at which the spatial density is increased because of the debris' population. It is characterized as critical, because objects need hundreds of years in order to decay and re-enter the atmosphere.
\item The second critical region is at the altitude 900-1000km, at which there is a high number of massive objects. Drag is also not sufficient in order to help objects decay and thus the spatial density increases over time.
\item The third critical region is at the altitude 1100-1300km. At this region the effect of drag is negligible, which means that any additional object builds up the orbital population.
%The future large constellations will be placed around these altitudes, which makes this region even more alarming. \cite{Somma 2019}
\end{itemize}



%(If I want to further analyze about the satellite viewing geometries and scanning patterns: Page 21,22 in http://www.atmosp.physics.utoronto.ca/people/strong/phy499/section2_05.pdf) --> It makes sense to do it if a use different ways to calculate the revisit time. So, if for every satellite I have the detailed info about the viewing geometry then I can do it. Another way is to write those info in the theory part and then calculate the revisit time, since I don't have much info ...

As far as the Space-Time Sampling in the LEO satellites: a) sampling is highly dependent on the orbit, b) area viewed on one orbit overlaps area viewed on the previous and succeeding orbits, c) usually view every point on the Earth twice a day, d) view each point a small number of local times but at varying elevation and azimuth angles.

%ABOUT REVISIT TIME:
%The irterval of time required for the satellite to complete its orbit cycle is not the same as the "revisit period". Due to the steerable sensors, the off-nadir angles, the large swath width , the revisit time can be less than the orbit cycle time. "The revisit period is an important consideration for a number of monitoring applications, especially when frequent imaging is required (for example, to monitor the spread of an oil spill, or the extent of flooding). In near-polar orbits, areas at high latitudes will be imaged more frequently than the equatorial zone due to the increasing overlap in adjacent swaths as the orbit paths come closer together near the poles." (Source: https://www.nrcan.gc.ca/maps-tools-and-publications/satellite-imagery-and-air-photos/remote-sensing-tutorials/acknowledgements-permission-use/9391)




\bigskip
\subsection{Earth Observation}
\bigskip

%Why EO? It has multiple capabilities
What do EO offers? "Optical imagery, high-revisit, all weather, and nighttime, Synthetic Aperture Radar, radio signal collection satellites that can geolocate signals emissions—essentially offering commercial electronic intelligence capabilities that can support transportation and logistics, emergency search and rescue, or spectrum mapping in addition to its existing applications for national security"
\cite{Hallex}

% TYPES OF EARTH OBSERVATION IMAGERY!!!!!!!!!
%https://business.esa.int/newcomers-earth-observation-guide

%----------------------------------------------------------------
%Regarding the field of remote sensing: "The critical design goal then is to place the satellite in an orbit that is low enough to permit a relatively short orbital period while at the same time the orbit is high enough to permit observation of a wide enough swath so that during a single orbit the Earth will rotate by less than the scan swath of the satellite's instrumentation. By placing a satellite at an altitude of about 850 km, you get an orbital period of roughly 100 minutes. At this altitude, you can get true global coverage if the scan swath of the satellite's instrumentation is about 3000 km." ABOUT THE SSO: "The orbit, however, can be improved if the orbital plane is inclined slightly away from a true N-S orbit. In this case, the asymmetric gravitational pull of the Earth introduces a slow precession in the orbital plane. With a inclination of about 98.7 degrees, the orbit will precess at almost exactly the same rate that the Earth rotates around the sun. That means that the satellite's orbital plane will appear to be fixed with respect to the sun, or sun-synchronous. Due to the inclination away from N-S, these satellites do not go directly over the poles, but do get close enough to provide true global coverages from a single satellite. Since the orbit is aligned with respect to the sun, in fact, you get twice daily coverage of every portion of the globe." Source: https://ral.ucar.edu/~djohnson/satellite/coverage.html#polar

%GEO KIND OF SUB-CLASSES:
%  Geosynchronous orbit:  is one in which the satellite orbits at the same angular velocity as the Earth.
%  There is also Geostationary orbit, which is a circular (zero eccentricity), geosynchronous orbit with an inclination equal to zero (equatorial). A satellite placed in such an orbit would appear, to an observer on the surface of the earth, to remain stationary in the sky. The footprint eventually depends on the field of view. Geometrically, a satellite at geostationary altitude can see a footprint about 70 degrees north and south of the equator. The benefits of such positioned satellite are increasing  its popularity and thus, crowded areas are produced in space. (Example about the benefit: A single sensor placed at a longitude equal to the center of US, which can monitor the weather pattern of the entire country continuously. Similarly, a communications relay station placed at a longitude in the middle of the Atlantic or Pacific oceans could allow direct communications between countries on either side or ships upon the seas. There are many more uses!)
%Because Earth rotates at a constant angular rate, a geostationary satellite must move at a constant speed in its orbit.
%Considering the fact that the semimajor axis of a geostationary orbit is around 42,168km, then it is calculated that the higher latitude that a geostationary satellite can view is λmax =81.3deg --> cos(λmax)= R_E / r_GEO = 6378/42,164 ==> λmax = 81.3deg. (WHY 42,164 when i said semimajor axis 42,168km ?)
%Some advantages and  disadvantages of geostationary orbits for remote sensing are the following. The Advantages are that almost all of a hemisphere can be viewed simultaneously and that the time evolution of phenomena can be observed. The disadvantages are that accurate measurements are difficult because the satellite is so far from Earth and that the polar regions are only observed at an oblique angle (good coverage only up to ~60° latitude).
%(Source: Campbell BA, McCandless W (1996) Introduction to space sciences and spacecraft applications. Elsevier)
%    Molniya Orbits: A Molniya orbit is an elongated elliptical orbit at an inclination of 63.4° for which the argument of perigee is fixed. Because ω is fixed, the apogee stays at a given latitude. These are used for communications satellites by the former Soviet Union because geostationary satellites provide poor coverage of the high latitudes of the FSU. They have also been suggested for meteorological observations at high latitudes. The apogee (and slowest speed) is over the FSU, and the perigee (and fastest speed) is over the opposite side of the globe, so that the satellite spends most of its time over the FSU. It stays in the apogge for about 8 hours!
\chapter{Application development}
\lhead{Chapter 3 \emph{Application development}}
\label{chap:3}
%\autoref{chap:3}
\bigskip

The limitation of the growing numbers of objects in space is nowadays an absolute necessity. The application, which is described below, has been developed towards this direction. More specifically, it can give the incentives to create space traffic management rules based on the orbit and capacity allocation, as well as the added value that each new satellite has.

The application has two major capabilities. The first one is the calculation of the revisit time of a single satellite or constellation (Section \ref{revisit time}). The second one is a tool, with which the user can find out what is the added value of an object based on the number of the satellites that are already in-orbit and have the same characteristics (Section \ref{added value}). This tool, in essence, has many more capabilities, which are presented in Chapter 4.%\ref{chap:4}.

\bigskip
\section{From satellite orbit definition to revisit time calculation}
\label{revisit time}
\bigskip

The implementation of this first part of the application was programmed and carried out in Python. In this stage, the user defines a satellite by providing its two-line elements (TLE). However, since the application is focused only on the Earth Observation satellites, parameters related to this field are also necessary to be given. Consequently, a number of steps need to be done in order to compute the orbit of the satellite: calculation of the polar coordinates in the orbital plane, the position that it has in the space-fixed system and later in the Earth-fixed system and finally the calculation of the longitude and latitude on the Earth's surface.
%Write epigrammatika all the different steps here.

\bigskip
\subsection{Necessary input data}
\label{input_data}
\bigskip

In this first step, the user can define a satellite by providing the TLE set to the application. Despite the fact that the classical orbital elements are used in the scientific community, it was decided that the TLE set is a more convenient and fast way of importing and handling the input data, since the latter way is applied universally and the orbit computation in the next step can be done without any further conversions of those given parameters.

From the elements of a TLE set, eight parameters are being extracted and used in the application. Those are the epoch date, inclination ($i$), Right Ascension of the Ascending Node (RAAN) ($\Omega$), eccentricity ($e$), argument of perigee ($\omega$), mean anomaly ($M$), mean motion ($n$) and the international designator, which is a unique code of a satellite. In the Figure \ref{tle}, it can be seen an example dataset with all the information that TLE carries.

%The epoch defines the time to which all of the time-varying fields in the TLE are referenced. In the first line of  the TLE, the last two digits of epoch year can be found in the columns 19-20 (in this case "09"), and the day of the year and fractional portion of the day for the epoch can be found in columns 21-32 (in this case "156.84140383"). In this TLE is referred to 19 May 2009.

More specifically, the epoch date is the number of days passed in the particular year. The information about the year is taken from the first two digits of the parameter and it is referred to below as epoch year. From the following digits of the parameter the exact month, day, epoch hour, minute and second can be found. Finally, the epoch time is calculated in seconds as:

\begin{equation}
\label{epoch}
\text{epoch day} = (\text{epoch hour} \cdot 3600) + (\text{epoch min} \cdot 60) + \text{epoch sec}.
\end{equation}

% If I want to elaborate about epoch, Julian date > check about Julian calculation: https://geotimedate.org/julian-time-converter/
% The epoch is given in Greenwich Mean Time (GMT). The epoch defines the time to which all of the time- varying fields in the set are referenced. (All times are measured in mean solar time). 

\begin{figure}
\centering
\includegraphics[width=0.9\textwidth]{Images/tle.png}\caption{An example two-line element (TLE) set. S is the sign of values, and E is of exponent. \textit{Source: \cite{Vallado}}}
\label{tle} 
\end{figure}

The parameters of eccentricity, inclination, RAAN, argument of perigee and mean anomaly can be seen graphically in the Figure \ref{keplerian_elements}. In short, eccentricity is a constant defining the shape of the orbit. There is a circular orbit when it is zero and an elliptical orbit when the number is less than one. To the given value of eccentricity from the TLE, a decimal at the beginning must be applied. As far as the inclination, which is given in degrees, it is the angle between the equatorial plane and the orbital plane. The RAAN or else the node (degrees) is the angle between vernal equinox and the point where the orbit crosses the equatorial plane and goes towards the north. Argument of perigee is the angle between the ascending node and the orbit's point of the closest approach to the Earth, which is called perigee. It is given in degrees. Finally, the mean anomaly is the angle of satellite location in the nominal orbit, which is referenced to a circular orbit with radius equal to the semi-major axis. It is measured from the perigee and it is also given in degrees. It should be also noted that the conversion of their units from degrees to radians is necessary, since many other computations will be followed \cite{Vallado}.

From the parameter of mean motion ($n$), the semi-major axis ($a$), as well as the period ($T$) of the orbit can be found. Namely, the mean motion is converted from the unit of $revolutions/day$ to $radian/sec$. Then, the semi-major axis is calculated from the Kepler's 3\textsuperscript{nd} law:
\begin{equation}
\label{3rd_keplers_law}
n^2 a^3 = \mu,
\end{equation}
%a = (\frac{\mu}{n^2})^{1/3},
with $\mu$ being the geocentric gravitational parameter.

\begin{figure}
\centering
\includegraphics[width=0.9\textwidth]{Images/keplerian_elements.png}\caption{The classical orbital elements: semi-major axis ($a$), eccentricty ($e$), inclination ($i$), Right Ascension of the Ascending Node (RAAN) ($\Omega$), argument of perigee ($\omega$), and true anomaly ($\nu$). \textit{Source: \cite{Vallado}}}
\label{keplerian_elements} 
\end{figure}

% Size and shape of orbit = a, e
% Orientation of orbital plane = i, Omega (rotation of orbital plane around Z-axis of geocentric equatorial coordinate system).
% Orientation of orbit in non-moving plane in space = omega == orientation of the elliptical orbit in the orbital plane == OR in which direction the perigee and apogee are pointing (Value 0-360 degrees)
% ALL THESE REMAIN UNCHANGED OVER TIME >> For the unperturbed 2-body problem!
% MEAN ANOMALY AT EPOCH: Define position of satellite in time in a TLE.

Apart from the parameters that were just mentioned, it is necessary to be added some parameters related to the Earth Observation sensor that the satellite has. The orbit lifetime, the type of sensor (see Section \ref{EO}) and the spatial resolution are some of them. However, the determining parameter in the calculation of the revisit time is the information of the swath width of the sensor, which is inserted in the units of kilometers. Swath is called the area of the Earth that is captured on the image and thus the bigger the number of swath is, the larger the area that is captured. In the Figure \ref{swath_width}, both the swath width and the spatial resolution are illustrated. Those parameters are usually inversely proportional; the higher spatial resolution, the less area is covered by a single image and thus the smaller the swath width. Finally, a parameter related to the viewing angle that the sensor has is the tilt angle. This angle designates whether the instrument points the Earth in a nadir direction or there is a pointing offset. 
% The swath width of the optical system is proportional to the orbit height. (Low altitude orbits have lack of coverage capability).

\begin{figure}
\centering
\includegraphics[width=0.9\textwidth]{Images/swath_width.png}\caption{Illustration showing the difference between swath width and spatial resolution. \textit{Source: Remote Sensing Technology Center of Japan - EO in Japan. Website: restec.or.jp (Accessed: October, 2020)}} %https://www.restec.or.jp/en/knowledge/sensing/sensing-3.html
\label{swath_width} 
\end{figure}

\bigskip
\subsection{Orbit computation}
\bigskip

The next step is the computation of the orbit. The imported parameters are given in a geocentric celestial reference system and they define a unique orbit.

\bigskip
\subsubsection{Polar coordinates in orbital plane}
\bigskip

Firstly, the polar coordinates, which are the radius ($r$) and the true anomaly ($v$) are calculated. For this calculation though, the eccentric anomaly ($E$) needs to be found through this transcendental Kepler's equation:
\begin{equation}
E - e \sin{E} = M,
\end{equation}

where $M$ is the mean anomaly. Once the equation is solved iteratively, the radius ($r$) and true anomaly ($v$) can be found.
\begin{equation}
r = a(1 - e \cos{E}),
\end{equation}

\begin{equation}
\tan{\frac{\nu}{2}} = \sqrt{\frac{1 + e}{1 - e}} \tan{\frac{E}{2}}.
\end{equation}

% If I want to elaborate more on how I found the true anomaly, check doc file (search for perigee passing time) or the same is written in the python code (before the actual commands - in the comments at the beginning). 

\bigskip
\subsubsection{Position \& velocity in space-fixed system}
\bigskip

The main idea of this task is the rotation of the orbital plane into the equatorial coordinate system. The angles that were used to the rotation matrices were the RAAN ($\Omega$), the inclination ($i$), and the argument of perigee ($\omega$). In the Figure \ref{orbit_to_space}, the rotation of the orbital plane is illustrated based on the aforementioned angles. For achieving that, the position (Equation \ref{position_orbital}) and velocity (Equation \ref{velocity_orbital}) in the orbital plane should be calculated first \cite{Montenbruck}:

\begin{figure}
\centering
\includegraphics[width=0.9\textwidth]{Images/orbit_to_space.png}\caption{The rotation of the orbital plane into the equatorial plane with the help of the following orbital elements of the satellite: inclination ($i$), RAAN ($\Omega$) and argument of perigee ($\omega$). \textit{Source: \cite{Montenbruck}}}
\label{orbit_to_space} 
\end{figure}

\begin{equation}
\label{position_orbital}
\vv{r_{b}} = r \begin{bmatrix} \cos{v} \\ \sin{v} \\ 0 \end{bmatrix}
\end{equation}
\begin{equation}
\label{velocity_orbital}
\dot{\vv{r_{b}}} = \sqrt{\frac{\mu}{a (1 - e^2)}} \begin{bmatrix} -\sin{v} \\ e + \cos{v} \\ 0 \end{bmatrix},
\end{equation}

where $\mu$ the geometric gravitational parameter (Equation \ref{3rd_keplers_law}).

Then, the position and the velocity in the space-fixed system are found through \cite{Montenbruck}:
\begin{equation}
\label{space-fixed}
\vv{r} = R_3 (-\Omega) R_1  (-i) R_3 (-\omega) \vv{r_{b}},
\end{equation}
\begin{equation}
\dot{\vv{r}} = R_3 (-\Omega) R_1  (-i) R_3 (-\omega) \dot{\vv{r_{b}}},
\end{equation}
where $\vv{r_{b}}$, $\dot{\vv{r_{b}}}$ are taken from the equations \ref{position_orbital}, \ref{velocity_orbital} respectively. The elementary matrices that were used are:
\begin{equation}
R_1(\alpha) = \begin{bmatrix} 1 & 0 & 0 \\ 0 & \cos{\alpha} & \sin{\alpha} \\ 0 & -\sin{\alpha} & \cos{\alpha} \end{bmatrix},
\end{equation}
for the rotation around the x-axis ($R_{1}$), and
\begin{equation}
\label{R3}
R_3(\alpha) = \begin{bmatrix} \cos{\alpha} & \sin{\alpha} & 0 \\ -\sin{\alpha} & \cos{\alpha} & 0 \\ 0 & 0 & 1 \end{bmatrix},
\end{equation}
for the rotation around z-axis ($R_{3}$) \cite{Montenbruck}.

\bigskip
\subsubsection{Position \& velocity in Earth-fixed system}
\bigskip

After the orbit calculation in the space-fixed system, the position and velocity in the Earth-fixed system can be found. Firstly the Julian date ($JD$) and the Julian centuries ($T_{UT1}$) need to be calculated by taking into account the epoch year, month and day, which are designated in the Equation \ref{Julian_date} as $YY, DD, MM$ respectively \cite{Vallado}. It should be noted that the Julian date is found at $0^{h}$ $0^{m}$ $0^{s}$ of the day, which means that for its calculation the epoch hour, min and sec were not used (Equation \ref{epoch}).

\begin{equation}
\label{Julian_date}
JD = 367 \cdot YY - \text{INT} \Bigg \{ \dfrac{7 \Big( YY + \dfrac{MM + 9}{12} \Big)}{4} \Bigg \} + \text{INT} \Big( \dfrac{275 \cdot MM}{9} \Big) + DD + 1721013.5
\end{equation}

\begin{equation}
\label{Julian_centuries}
T_{UT1} = \frac{JD - 2451545}{36525}
\end{equation}

Since the Julian centuries has been found (Equation \ref{Julian_centuries}), the Greenwich mean sidereal time at midnight can be calculated in seconds \cite{Vallado}:

\begin{equation}
\label{sidereal_angle}
\resizebox{\textwidth}{!}
{
$\textit{sidereal angle}_{\text{GMST 0h}} = 24110.54841 + 8640184.812866 \cdot T_{UT1} + 0.093104 \cdot T_{UT1}^{2} - 6.2 \cdot 10^{-6} \cdot T_{UT1}^{3}$
}
\end{equation}

Based on the rotation rate of the Earth ($\Omega_{E}$) and the sidereal angle (Equation \ref{sidereal_angle}), the angle of rotation ($\theta_{0}$) between the space and Earth system is calculated (Equation \ref{theta_angle}). The parameter $t$ is the universal time UT1 in seconds past $0^{h}$.

\begin{equation}
\label{rotation_rate}
\Omega_{E} = \frac{2 \pi}{86164}
\end{equation}

%% What is the 86164 ? It is the length of the SIDEREAL DAY, which is defined as the time it takes earth to complete one revolution around its own axis as measured by observing a distant star. It is 86164sec long.
%% On the other hand, the SOLAR DAY is a bit more 86400sec long, because it is the time that the satellite needs to roatate from its axis but also to line up with the sun-earth line, since it also orbits the sun.

\begin{equation}
\label{theta_angle}
\theta_{0}(t) = \Omega_{E} \cdot t + \textit{sidereal angle}_{\text{GMST 0h}}
\end{equation}

Both angles, the angle of rotation $\theta_{0}$ and the sidereal angle, should be in radian in order to be used in the following steps. Finally, the position in the Earth-fixed system is calculated using the rotation matrix around the z-axis (Equation \ref{R3}) as:
\begin{equation}
\vv{r}_{Earth-fixed}(t) = R_{3}(\theta_{0}(t)) \vv{r},
\end{equation}
where $\vv{r}$ is the position in the space-fixed system (Equation \ref{space-fixed}).

\bigskip
\subsubsection{Longitude \& latitude on the Earth's surface}
\bigskip

In order to calculate the position of the sub-satellite's points on the Earth's surface, the longitude and latitude is needed. In this application the geocentric latitude is computed as a simplification assumption. The equations that link the Earth-fixed coordinates with the $\lambda$ and $\phi$ are the following:
%People often use the geodetic latitude (flattened Earth), however it would probably not make much difference in this application.

\begin{equation}
\tan{\lambda} = \frac{y_{\text{Earth-fixed}}}{x_{\text{Earth-fixed}}},
\end{equation}

\begin{equation}
\tan{\phi} = \frac{z_{\text{Earth-fixed}}}{\sqrt{x^2_{\text{Earth-fixed}} + y^2_{\text{Earth-fixed}}}}.
\end{equation}

The groundtrack of the orbit, which is the path of the sub-satellite point as the satellite travels through its orbit, can be consequently acquired. However, the actual groundtrack of a satellite differs from a simple circle that results from the intersection of the orbital plane with the surface of the Earth. For a satellite with a period $T$, the longitude $\lambda$ is shifted from one revolution to the next \cite{Montenbruck} :
\begin{equation}
\label{delta_lambda}
\Delta \lambda_{\Omega} = - \Omega_{E} \cdot T = - 0.2507^{\circ}/ min \cdot T.
\end{equation}

This is due to the Earth's rotation, as it can be seen in the Figure \ref{groundtrack-fixed-rotating}.

% Check from Vallado p.898 the "groundtrack shift".
% Another way of finding the \Delta \lambda is: \Delta \lambda = (360 (degrees) * P[min] (period))/ 1440.
% A low spatial resolution indicates a high temporal resolution and the opposite a high spatial indicates a low temporal resolution.

\begin{figure}
\centering
\includegraphics[width=0.9\textwidth]{Images/groundtrack-fixed-rotating.png}\caption{Illustration of the groundtracks in cases where the Earth is fixed and rotating. \textit{Source: \cite{Vallado}}}
\label{groundtrack-fixed-rotating} 
\end{figure}

% Reference: Vallado p. 147

\bigskip
\subsubsection{Perturbation on the orbital elements}
\bigskip

As a first approximation, it is reasonable to treat Earth as spherically with constant mass density. However, since this software requires long-term orbit propagations in order to produce the desired outcome of the revisit time, a better approximation of the Earth should be considered.

In reality, Earth is not exactly spherical, but oblate, which affects the orbit calculations as described below. 
Additionally, the focus of this application is on the low Earth orbit (LEO) satellites, to which the gravity field is the largest perturbation causing secular motion to some of the orbital elements. In this work only the first-order zonal harmonics of J2, accounting for the Earth's flattening, are used. The higher order terms are omitted, since they are a few orders of magnitude lower.

One of the affected orbital elements by the zonal harmonics of J2 is RAAN. This means that the orbital plane itself rotates around Earth's N-S axis. The rate of precession is also called nodal regression and it is given by:
\begin{equation}
\dot{\Omega} = - \frac{3 n R_{\bigoplus}^{2} J_{2} \cos{i}}{2 a^{2}},
\end{equation}
where $R_{\bigoplus}$ is the equatorial radius of the Earth according to WGS-84 (6378.137 km), and as $n$ the mean motion is used \cite{Montenbruck}.

Due to secular perturbations, the argument of perigee is also affected. The rate of precession is also called the apsidal rotation and it is given by:
\begin{equation}
\dot{\omega} = - \frac{3 n R_{\bigoplus}^{2} J_{2}}{4 a^{2}} (1 - 5 \cos^{2}{i}).
\end{equation}

% Vallado p. 147, 674 + (Repeat groundtrack Ch.11 Section 9.6.1) +  Montenbruck p.50
% If you want to elaborate, check the doc notes about "Perturbing forces on orbital elements."

\bigskip
\subsubsection{Revisit time definition}
\bigskip

The goal of the first part of the application is to find the revisit time of the satellite. Thus, its definition needs to be noted. The revisit time is the time elapsed between two successive observations of the same ground point on the surface of the Earth \cite{Luo}. To be more specific, since this application will produce one number as the revisit time of a satellite, this number refers to the average revisit time of all the ground points in the equatorial plane. However, the user has the capability of inserting in the application a latitude, in which he/she is interested in knowing the revisit time, other than the one in the equator which comes by default. More elaborated description regarding its calculation is being followed.

%ABOUT REVISIT TIME:
%The interval of time required for the satellite to complete its orbit cycle is not the same as the "revisit period". Due to the steerable sensors, the off-nadir angles, the large swath width , the revisit time can be less than the orbit cycle time. "The revisit period is an important consideration for a number of monitoring applications, especially when frequent imaging is required (for example, to monitor the spread of an oil spill, or the extent of flooding). In near-polar orbits, areas at high latitudes will be imaged more frequently than the equatorial zone due to the increasing overlap in adjacent swaths as the orbit paths come closer together near the poles." (Source: https://www.nrcan.gc.ca/maps-tools-and-publications/satellite-imagery-and-air-photos/remote-sensing-tutorials/acknowledgements-permission-use/9391)

\bigskip
\subsection{Computational calculation of revisit time}
\bigskip

One key element needed in order to calculate the revisit time is the observation swath width, or the spatial extent or else the area that is captured on the ground by the satellite's sensor. This is an information that is already given, as stated in the Section \ref{input_data}, and this is the parameter of the swath width (see Figure \ref{swath_width}). Thus, taking into account the swath width, a grid in the global map can be created, which will be consisted of cells simulating the area on the ground that is captured by the satellite. The global map is represented as a two-dimensional map in comparison with a three-dimensional map, since the computations can be executed easier later.

\bigskip
\subsubsection{Map projection}
\bigskip

The map projection that was chosen is a equidistant cylindrical, which forms a grid of vertical and horizontal lines of constant spacing. In this projection the poles are represented as straight lines across the top and bottom of the grid, the same length as the equator line. It is worth mentioning that the distortion in the shape, scale and area increases as the distance from the equator increases \cite{Lapaine}. An example of this map projection is shown in Figure \ref{map_projection}, as it was generated by the code written in Python 3.8.5\footnote{\label{Python_packages}\textit{In general, the Python packages used in this work are: numpy, math, matplotlib, datetime, configparser, seaborn, pandas, cartopy.crs \& psycopg2.}}. The selection of this projection was made, since the distortions in the polar regions can be easily corrected, in comparison i.e. with a Mercator projection. The grid of equal rectangles resembles also an array in python, which makes the calculations more comprehensible.

\begin{figure}
\centering
\includegraphics[width=0.9\textwidth]{Images/map_projection.png}\caption{A cylindrical equidistant map projection that was used in the application. It forms a grid of equal rectangles.}
\label{map_projection}
\end{figure}

\bigskip
\subsubsection{Grid creation}
\bigskip

The grid that is created on the global map consists of quadratic cells, whose edge length corresponds to the swath width. Since the global map handles geographic degrees, 360 degrees in the longitude and 180 degrees in the latitude, the swath width needs to be converted as it was given originally in the unit of kilometers. This conversion is based on the circumference of the Earth, which is roughly 40,000 km, divided by 360 degrees. As a rule of thumb, 111 kilometers are approximately 1 degree and for simplification reasons in the Equation \ref{grid_interval} the conversion rate of 0.008 is used. Thus, the interval between the cells, which is also named as grid interval in this application, can be found from the equation:
\begin{equation}
\label{grid_interval}
\textit{grid interval} \text{ [degrees]} = \textit{swath width} \text{ [km]} \cdot 0.008
\end{equation}

Considering that the grid interval is known (Equation \ref{grid_interval}), the number of the cells that can fit in the global map is found. It should be noted that some of the instruments in a satellite, have more than one swath modes with whose characteristics the images are captured. As an example, the constellation Sentinel-1 can be mentioned, whose satellites' sensors have the following modes: Extra Wide Swath Mode (EW), Interferometric Wide Swath Mode (IW), Stripmap Mode (SM) and Wave Mode (WV) \footnote{\label{Sentinel1_source}\textit{Available at: https://sentinel.esa.int/web/sentinel/user-guides/sentinel-1-sar/revisit-and-coverage (Accessed November 20, 2020)}}. The default approach of the software is to calculate the revisit time taking as swath width the larger of the available ones. In this way, considering the fact that the wider the swath is, the larger the area is captured, then the revisit time of the satellite might also be more frequent. However, the user can find out the revisit time of the satellite using the swath mode of his/her interest.

An important feature in the creation of the grid is that the zero degree latitude and longitude is always included in a cell (Figure \ref{map_projection_0covered}). In the case of a satellite with zero inclination, whose viewing latitudes are mostly the equatorial ones, cells containing the zero degree latitude are the ones apparently revealing the information about the revisit time in those regions. So in those cases, cells including the zero latitude are of utmost importance. However, this application focuses on Earth Observation (EO) satellites in the LEO region, whose orbits are usually near-polar. Hence, this feature can be truly beneficial when equatorial or tropical orbits are imported to the application.

\begin{figure}
\centering
\includegraphics[width=0.9\textwidth]{Images/map_projection_0covered.png}\caption{The zero longitude and latitude is always centered to a grid's cell. Taking as an example a 100 km swath width sensor, the cell size is computed as 0.8 (Equation \ref{grid_interval}), which results in the central cell ranging from -0.4 to +0.4 degrees in longitude and latitude.}
\label{map_projection_0covered}
\end{figure}

\bigskip
\subsubsection{Correction of map distortion}
\label{Correction of map distortion}
\bigskip

As it was mentioned before, the map projection that is used, exhibits distortions, which increase the further the place is located from the equator. This happens due to the fact that the poles are represented as straight lines, which are parallel and have the same length as the equatorial line. An example of how the swath width of a satellite seems to increase as the satellite moves towards the poles can be seen in Figure \ref{distortion_show-swath-width}. All the same, the swath width of the satellite's sensor remain constant throughout its orbit.

\begin{figure}
\centering
\includegraphics[width=0.9\textwidth]{Images/distortion_show-swath-width.png}\caption{A near-polar orbit illustration of a sun-synchronous meteorological satellite. Its instrument scans approximately 3000 km wide swath. Due to the map distortion, the viewing area seems to be larger when the satellite passes closer to the poles. \textit{Source: National Center for Atmospheric Research, Boulder, Colorado, USA. Url: ral.ucar.edu/~djohnson/satellite/coverage.html (Accessed September 30, 2020)}}
\label{distortion_show-swath-width}
\end{figure}

For the correction due to the map distortion, the simple assumption of a spherical Earth has been made. Then, it has been assumed, that the equator's length is 360 degrees, whereas the meridians of the Earth intersect on the poles, leading to a point (zero length) in the poles. As it can be seen in Figure \ref{correction_map_distortion}, the hypotenuse of the illustrated triangle is found using the Pythagorean theorem, and based on this side, the angle $\phi$ is found from: $$ \cos{\phi} = \frac{90}{\text{hypotenuse}}.$$
Thus, based on the $\phi$ angle (constant) and the latitude of the sub-satellite point, a swath width depicting the area that the satellite is actually scans, is calculated. In this way, the distortion is corrected.

\begin{figure}
\centering
\includegraphics[width=0.9\textwidth]{Images/correction_map_distortion.png}\caption{The triangle shows the approximation method of correcting the distortion of the map. The poles instead of having a 360 degrees length, are assumed to be points.}
\label{correction_map_distortion}
\end{figure}

\bigskip
\subsubsection{Space-time sampling of the satellite}
\bigskip

Before the actual procedure of counting cells starts, it should be assured that the interval between the sub-satellite points will give enough information for the revisit time calculation. To be more specific, during the orbit that is depicted in the groundtrack, there should not be empty cells neither too many points on the same area, so that it seems that the satellite has passed more than once. This so-called space-time sampling in the LEO satellites is dependent on the orbit. For this reason, the height of the satellite should be considered. The height of the satellite is given through the semi-major axis $a$, which is connected with the mean motion $n$ through the Equation \ref{3rd_keplers_law}. From the equation: $$n = \frac{2 \pi}{T}, $$
the information of the period $T$ is found, with which the interval between the points can be calculated. It is a fact that in one orbit (one period) the satellite passes above 360 degrees in longitude, as it can be seen in Figure \ref{ascending-descending} showing the groundtrack of Sentinel-2A. Then, the unknown time interval can be found through the equation:
\begin{equation}
\textit{interval} = \frac{\textit{grid interval} \cdot T}{360 \text{ [degrees]}}
\end{equation}

since the grid interval is known (Equation \ref{grid_interval}).

% Change how to find the interval: for LEO the Δλ is about 340 degrees and for a retrograde orbit the Δλ is about 380 degrees! I used 360 degrees (for everything) \cite{Vallado}. Source Vallado p.149

\bigskip
\subsubsection{Tracking of satellite's path}
\bigskip

The calculation of the revisit time is actualized by counting the cells that the satellite has passed. Additionally to this, the time of the passage is tracked so that the time elapsed between two successive observations of the same cell can later be calculated. 

In this tracking, the four corners of the final image are taken into account. This is necessary for the reason that the area being scanned by the sensor is dependent on the latitude of the satellite's location (see beginning of section \ref{Correction of map distortion}). It should be also noted that a cell is counted as a cell captured by the satellite only when the final image taken by the sensor covers more than 50\% of the area of the cell.

As it can be seen in Figure \ref{map_projection_0covered}, since the zero degree longitude and latitude is always centered on a cell, there are cells on the borders of the map, which are cut. However, a mindful calculation of the cell counting even in the borders of the map is guaranteed in this application. When a sub-satellite point is located in a border cell, as in Figure \ref{correct_calculation}, and some corners might be located in the part of the globe that is appeared on the opposite side of the two-dimensional map projection, then the counted cells are all those, which have been captured at least by 50\% by the sensor. Those cells in the example case of Figure \ref{correct_calculation}, have been colored.

\begin{figure}
\centering
\includegraphics[width=0.9\textwidth]{Images/correct_calculation.png}\caption{Example of sub-satellite point on Earth and the four corners of the final captured image by its sensor.}
\label{correct_calculation}
\end{figure}

It also needs to be mentioned that the coverage area along the equator is dependent not only on the swath width, but also on the orbit's inclination, which fact is illustrated in Figure \ref{swath_width_groundtrack}. Due to the fact that this software is focused on EO satellites, which are mostly in near-polar orbits, and thus with inclination close to 90, the coverage along the equator is assumed to have the same value as the swath width.

\begin{figure}
\centering
\includegraphics[width=0.9\textwidth]{Images/swath_width_groundtrack.png}\caption{Illustration showing the dependence between the coverage area along the equator and the parameters of swath width and inclination. \textit{Source: \cite{Elachi}}}
\label{swath_width_groundtrack}
\end{figure}
% http://www.its.caltech.edu/~ee157/ >> VERY NICE INFORMATIVE ppt. There are plenty other topics in Remote Sensing! The one I took the figure is "Orbit Mechanics".

\bigskip
\subsubsection{Tracking modification depending on sensor type}
\bigskip

One determinant parameter of the revisit time calculation is the swath width of the sensor as it was analyzed above. Nevertheless, the sensor type can change the final calculation. This is for example whether the instrument belongs to the passive or active imagery sensors (see section \ref{EO}). The reason behind is that when a satellite has passive instrument on board, it needs to be on the sunlit side of the Earth since those sensors record the reflected solar energy. Most of the EO satellites today are in near-polar sun-synchronous orbits, which have the following characteristic. The descending pass of the satellite, which is when the satellite travels southwards, is usually on the sunlit side in the case of a sun-synchronous orbit \cite{Kramer 2002}. For this reason, it is assumed that all the sun-synchronous EO satellites used in this software are on the sunlit side during the descending pass of their orbit.

This application, therefore, checks whether the satellite has a passive sensor or not, and in case it has, the revisit time is calculated only for the descending passes of the satellite. In the Figure \ref{ascending-descending}, one orbit of the Sentinel-2A is shown; the mean local solar time (MLST) at the descending node is 10:30 (am) \footnote{\label{MLST_Sentinel2}\textit{Available at: https://sentinel.esa.int/web/sentinel/missions/sentinel-2/satellite-description/orbit (Accessed December 1, 2020)}}, which means that the descending pass is on the sunlit side of the Earth and the ascending on the shadowed.

\begin{figure}
\centering
\includegraphics[width=0.9\textwidth]{Images/ascending-descending.png}\caption{The groundtrack of Sentinel-2A orbiting once the Earth. The ascending and descending pass can also be seen.}
\label{ascending-descending}
\end{figure}

Yet another parameter that can change the calculation of the revisit time is whether the instrument has a nadir or tilted looking sensor (see Figures \ref{nadir_looking} and \ref{tilted_looking}). In case a sensor has a tilted looking capability, the result of the revisit time will not change, since all the counting of the cells will be shifted perpendicular to the groundtrack direction. However, if the information about the viewing geometry of the sensor is available, then the calculation of the revisit time can be actualized having taken into account this parameter as well.

\begin{figure}
\centering
\includegraphics[width=0.9\textwidth]{Images/nadir_looking.png}\caption{An illustration of a nadir looking sensor. The satellite ($S$) is directly beneath the nadir point. Regarding the rest of the symbols: $P$ depicts the point of interest, $h_{\text{ellp}}$ the altitude of the ellipse, $\eta_{\text{FOV}}$ the boresight angle of the field of view, and $\Lambda$ the ground range. \textit{Source: modified \cite{Vallado}}}
\label{nadir_looking}
\end{figure}

\begin{figure}
\centering
\includegraphics[width=0.9\textwidth]{Images/tilted_looking.png}\caption{An illustration of a tilted looking sensor. The $\eta_{\text{center}}$ depicts the tilt angle. Regarding the rest of the symbols, see Figure \ref{nadir_looking}. \textit{Source: modified \cite{Vallado}}}
\label{tilted_looking}
\end{figure}

\bigskip
\subsubsection{Final calculations of revisit time}
\bigskip

The final calculations of the combined revisit time along the equator or along a line of latitude of interest to the user are described below. As it was mentioned above, the longitude cells have been created based on the swath width, whose length was found from the Equation \ref{grid_interval}. Whenever the satellite passes from each of those cells, the time of passage is saved. Then, based on those timestamps, the time elapsed between two successive observations is calculated for every longitude cell individually. In other words, the revisit rate between the successive passages from every longitude cell is found and then the average of those.

Finally, in order to calculate the combined revisit time at the equator or at a specific latitude, the longitude cells that have not been passed from the satellite are not taken into account. The average of the revisit rates that have been found from all the rest longitude cells that the satellite has passed, is calculated creating the final result. The reason why the longitude cells, from which the satellite has not passed, are not taken into consideration is because the final result of the revisit time will be inaccurate, since it will be also affected by these cells, which seem to have zero average revisit rate, whereas it is zero due to the fact that the satellite never passed. \footnote{\label{Flock_source}\textit{It should be mentioned that for the realization of those calculations, lists of predetermined length filled with zeros are created beforehand for computational reasons. This is the reason why the cells, from which the satellite has not passed, have zero value in the list in which the average revisit time of every longitude cell along the equator or a line of latitude is stored.}}


\bigskip
\subsubsection{Additional functions of the application}
\bigskip
The user has the option of selecting a certain latitude, in which he/she is interested in knowing the revisit time. Then, the above explained procedure of the calculation of the revisit time is followed again for the requested latitude. As already mentioned, the default option is the calculation of the revisit time at the equator.

Another capability of this application is that the revisit time in any latitude of not only one satellite, but of a constellation can be found. This means that the added value of the dozens of new satellites that are added to big EO constellations can be investigated.

\bigskip
The python code can be found in the Appendix (Section \ref{app}).

\bigskip
\section{Added value of satellite based on operationally similar objects}
\label{added value}
\bigskip

The second capability of the application is to provide the user with the information about the current state of the LEO region in the field of EO satellites. More specifically, according to the user's request, the application parses, analyzes or calculates the orbital characteristics and attributes of the EO satellites that have been analyzed previously by the application (see section \ref{revisit time}) based on the input parameters that were imported by the user. Some real examples are presented in the next chapter.

\bigskip
\subsection{Database}
\label{database_ch3}
\bigskip

For the actualization of this goal, the creation of a database is necessary. In this way, there will be a system, in which all the information related to EO satellites orbiting in the LEO region can be stored along with the results obtained from the application that was previously described. Then, not only a user can benefit by finding out what is the added value of a future satellite based on how many other similar objects offer this service already, but also this database can be the springboard to new research studies related to the space traffic management rules. These rules need to be implemented based on the orbit and capacity allocation.

\bigskip
The database was implemented in \textit{PostgreSQL}, which is an open-source relational database management system with great capabilities.  

The main table that was created contains the keplerian elements of an orbit, which can be found in the TLE set of a satellite. Additionally, the parameters related to the EO sensors and instruments are also added (see section \ref{input_data}). In a nutshell, the added elements are the following: the unique international designator of a satellite, its name, the epoch, inclination ($i$), RAAN ($\Omega$), eccentricity ($e$), argument of perigee ($\omega$), mean anomaly ($M$), mean motion ($n$), the date of the launch, operational lifetime, the name of the sensors on board, swath width, spatial resolution, and the type of instrument, whether it is an active or passive sensor.

The information that is also plugged in to the table is the satellite's revisit time in the equator. In order to be actualized, the application that was described in the section \ref{revisit time} runs having as input data, the information of this table. Then, the result of the application, which is the revisit time, is imported as a new information to the database.

\bigskip
\subsubsection{Sources of data}
\bigskip

The sources of the data are numerous. For this reason, another table containing this information was created. This is necessary in the case of an update in the information or when a parameter needs to be checked. Some of the most frequent references regarding the TLE set is the \textit{CelesTrak} from \textit{NORAD} and the \textit{Space-track.org} webpages. There are also multiple databases, which include in their records relevant to this application data.

\bigskip
\subsection{Classification of Earth Observation Field}
\bigskip

It has been noticed that in the existed databases, like ESA's DISCOS, there is only a very coarse classification available or none at all. This lack of distinction between the EO missions makes the search of added value of a satellite even harder. This happens due to the fact that the main objective of the mission is missing as an information from the existing databases. For this reason, a classification of EO field was made.

The categorization of the EO field is consisted of two main layers. The first one, which is the most general, refers to the source of funding of the mission as well as to the target group, which will use the satellite's output. So, in this layer the options are either commercial, research, defense, or civil (Figure \ref{classification_1st_layer}).

\begin{figure}
\centering
\includegraphics[width=0.9\textwidth]{Images/classification_1st_layer.png}\caption{The first layer of the classification of the field of Earth Observation.}
\label{classification_1st_layer}
\end{figure}

The second layer refers to the main objective of the mission. This can be related to environmental monitoring, emergency management, science, or security. These two aforementioned layers are not interconnected. This means that a satellite can have attributes with any possible combination of those. The rest of the layers are connected to the second one as it can be seen in Figure \ref{classification_2nd_layer}. In other words, from the main objective of a mission (second layer), a more explanatory and detailed description of the mission's goal is given in those labels.

\begin{figure}[!htb]
\centering
\includegraphics[width=0.9\textwidth]{Images/classification_2nd_layer.png}\caption{The second layer of the EO classification together with its further sub-categories.}
\label{classification_2nd_layer}
\end{figure}

This classification was implemented by taking into account various mission objectives and characteristics that were found in \cite{Newspace} and other databases, as it is mentioned in the previous section. Another source of influence was the work of Kramer \cite{Kramer 2002}. All in all, the creation of an objective, accurate, general and at the same time specific classification is a project that can have many solutions.

\bigskip
\subsection{Data import to the database}
\bigskip

The application as it was described at the beginning of this chapter together with the advantages and capabilities that a database brings, create a useful tool. This can be further improved by inserting more data to the database.

There are three possible ways of importing data in the database. The first one is through the server of the PostgreSQL locally, or in other words via the SQL application. The second one is through a configuration file (.ini file). In this way, the application first examines whether the imported satellite already exists in the database, or there is just a missing information regarding its revisit time, and then according to the answer it inserts it or not. The third way of importing data is through a python script (see section \ref{app}), which is an automated process of inserting the TLE elements in the database.

% In the first way, I also need to import manually not only the attributes, but also all the parameters related to the "classification" and the "sources" table.

\bigskip
The tables of the created database can be found in the Appendix (section \ref{app}).
\chapter{Experiments \& Results}
\lhead{Chapter 4 \emph{Experiments \& Results}}
\label{chap:4}
%\autoref{cha:4}

\bigskip

In this chapter the results of the revisit time calculation are the first to be presented. Then, several experiments are demonstrated, which follow the logical sequence of a possible application use. Namely, examples of user's requests that ask to find out if there are currently operational satellites that have some specific properties and to name them, are shown. Then, based on the application's answer, the user can opt to find the revisit time of a group of satellites or of a constellation. Thus, several experiments related to the calculation of the collective revisit time are demonstrated.

Additionally, a user has the possibility to find the position of a prospective new satellite, which can achieve together with a group of satellites already in orbit a more frequent revisit time. It also needs to be mentioned that the software calculates the revisit time specified in terms of the latitude. As a default approach the revisit time at the equator is found, however the software can calculate the revisit time at any latitude of interest to the user.

It also needs to be mentioned that for the experiments and results that are presented, 300 orbits has been selected to be the duration of the simulations. Since the average period of the experimented satellites and the EO satellites in LEO that have been plugged in the database (Section \ref{database_ch3}) is approximately 100 minutes, then in 300 orbits, 20 days have elapsed. Based on the revisit rates of satellites with similar characteristics, 20 days is a period in which the revisit time of those satellites can be acquired \cite{Christopherson}. Nevertheless, for satellites, which fall into a different category than of LEO or EO and orbit for example in higher altitudes, this assumption needs to be changed.

\bigskip
\section{Revisit time calculation of single satellites}
\bigskip

As it was mentioned in Chapter \ref{added value}, the creation of a database, which contains information about the orbit and attributes of the on-board sensors of LEO satellites in the field of EO, is of great importance for assessing the added value of a planned EO mission. For this reason, the first results to be presented show the revisit time of satellites, which are hosted in the database. Nevertheless, since the application needs also to be tested (Chapter \ref{chap:5}), the results of renowned and well documented satellites, like the Sentinels were chosen to be presented.

The first example to be showed is the satellite Sentinel-2A. In the Figure \ref{revisit_time_Sentinel2A} the average revisit time per longitude cell along the equator can be found. The length of the longitude cell was determined based on the swath width of the satellite's sensor (Equation \ref{grid_interval}), which in this case is 290 km \footnote{\label{Sentinel-2}\textit{Available at: https://sentinel.esa.int/web/sentinel/missions/sentinel-2 (Accessed November 20, 2020)}}. As it can also be seen in the figure, there are some longitude cells which the satellite does not pass. After having computed the average revisit time per longitude cell, the average revisit time at equator is found, which is depicted in blue line in the figure. As shown, the average revisit time at equator of Sentinel-2A is around 10 days (Figure \ref{revisit_time_Sentinel2A}).

\begin{figure}
\centering
\includegraphics[width=0.9\textwidth]{Images/revisit_time_of_SENTINEL-2A.png}\caption{Average revisit time per longitude cell along the equator. The length of the longitude cell was determined based on the swath width of the satellite's sensor (Equation \ref{grid_interval}). The average revisit time at the equator of Sentinel-2A is 9.66 days.}
\label{revisit_time_Sentinel2A}
\end{figure}
% Run the code main_code_db_aggregated.py

An equivalent graph showing the average revisit time at equator of Sentinel 3A \footnote{\label{Sentinel3A_source}\textit{Available at: https://sentinel.esa.int/web/sentinel/user-guides/sentinel-3-synergy/coverage (Accessed November 20, 2020)}} can be found in Figure \ref{revisit_time_Sentinel3A}. The average revisit time at the equator for this satellite is less than 2 days, since the swath width of its optical instrument is as large as 1250 km. For validation purposes, two other graphs showing the average revisit time of satellites are presented. In the Figure \ref{revisit_time_of_ALSAT 1B}, the revisit time of the Algerian Alsat-1B satellite can be found, whereas in Figure \ref{revisit_time_of_GOMX-4A} the one of the Danish GOMX-4A satellite \cite{Christopherson}.

\begin{figure}
\centering
\includegraphics[width=0.9\textwidth]{Images/revisit_time_of_SENTINEL-3A.png}\caption{Average revisit time per longitude cell along the equator. The average revisit time at the equator of Sentinel-3A is 1.22 days.}
\label{revisit_time_Sentinel3A}
\end{figure}
% Run the code main_code_db_aggregated.py

\begin{figure}
\centering
\includegraphics[width=0.9\textwidth]{Images/revisit_time_of_ALSAT_1B.png}\caption{Average revisit time per longitude cell along the equator. The length of the longitude cell was determined based on the swath width of the satellite's sensor (Equation \ref{grid_interval}). The average revisit time at the equator of Alsat-1B is 5.31 days.}
\label{revisit_time_of_ALSAT 1B}
\end{figure}
% Run the code main_code_db_aggregated.py

\begin{figure}
\centering
\includegraphics[width=0.9\textwidth]{Images/revisit_time_of_GOMX-4A.png}
\caption{Average revisit time per longitude cell along the equator. The average revisit time at the equator of GOMX-4A is 4.77 days.}
\label{revisit_time_of_GOMX-4A}
\end{figure}
% Run the code main_code_db_aggregated.py

\bigskip
\section{Request to find operational satellites with certain \\properties}
\bigskip

Another capability that this application brings is that the user can request to find whether there are currently operational satellites, which have certain characteristics or offer specific properties that are of the user's interest. The properties that a user can specify, are related to the various fields of attributes that were added to the database.

To be more exact, these properties can pertain to the date that a satellite was launched, its operational lifetime, the type of sensors that it has on board, and the resolution that they can achieve, as well as the type and the objective of the mission, such as whether it is a commercial, civil, defense or research mission and the main goal and focus of the mission (Figure \ref{classification_1st_layer}, \ref{classification_2nd_layer}). All the above information about the satellites, which have been added to the database, can be found in the tables \ref{table:1}, \ref{table:2}, \ref{table:3} in the section of Appendix. Moreover, the user can find out whether a satellite, which has a specific desirable revisit time at the equator exists. This information of the revisit time has been added to the database after calculating it for all the elements of the database based on the procedure that was described in Chapter \ref{chap:3}.

\bigskip

The first experiment to be presented is of a user, who wants to find which are the satellites that have an active sensor, have less than six days of revisit time at the equator and have more than three years of operational lifetime. In the algorithm \ref{Request1}, the whole procedure can be found, together with the output that the code produced. As a second experiment, a user wants to find which are the satellites that have as their main objective the monitor of the land ("Land monitoring") and the resolution of their sensors is four meters (Algorithm \ref{Request2}).

\begin{algorithm}[H] % When you have [H], then the algorithm appears exactly under the text. If you don't write it then, it goes at the end of the whole pdf.
\caption{Request regarding sensor type, revisit time \& operational lifetime.}\label{Request1}
\hspace*{\algorithmicindent} \textbf{Input: }\\
\hspace*{\algorithmicindent} $\textcolor{purple}{sensor } \gets \text{'active'}$ \\
\hspace*{\algorithmicindent} $\textcolor{purple}{days } \gets \text{6}$ \\
\hspace*{\algorithmicindent} $\textcolor{purple}{lifetime } \gets \text{'3 years'}$
\begin{algorithmic}[1]
\Procedure{}{}
\State $\text{Connection to the database}$ 
\State $\textbf{SELECT } \textit{name }$ \Comment{The output will be the name of the satellites}
\State $\textbf{FROM } \textit{attributes }$ \Comment{\textit{attributes}: main table of the database}
\State $ \textbf{WHERE:} $
\State \hspace*{\algorithmicindent} $\textit{type_of_sensor = } \text{ } \textcolor{purple}{sensor} \textbf{ AND} $
\State \hspace*{\algorithmicindent} $\textit{revisit_time < } \text{ } \textcolor{purple}{days} \textbf{ AND}$
\State \hspace*{\algorithmicindent} $\textit{(now() - } \textbf{CAST}\textit{(date_launched } \textbf{AS TIMESTAMP) <  } \text{ } \textcolor{purple}{lifetime} \text{ ;}$
\EndProcedure
\Comment{\textit{now()}: function showing current date}
\end{algorithmic}
\hspace*{\algorithmicindent} \textbf{Output:}\\ \hspace*{\algorithmicindent} \textit{[Sentinel-3B, Sentinel-6, SAOCOM-1A, NOVASAR-1, RCM-1, RCM-2, RCM-3]}
\end{algorithm}

\begin{algorithm}[H] % When you have [H], then the algorithm appears exactly under the text. If you don't write it then, it goes at the end of the whole pdf.
\caption{Request regarding mission objective \& sensor resolution.}\label{Request2}
\hspace*{\algorithmicindent} \textbf{Input: }\\
\hspace*{\algorithmicindent} $\textcolor{purple}{objective } \gets \text{'land monitoring'}$ \\
\hspace*{\algorithmicindent} $\textcolor{purple}{meters } \gets \text{4}$
\begin{algorithmic}[1]
\Procedure{}{}
\State $\text{Connection to the database}$ 
\State $\textbf{SELECT } \textit{attributes.name, attributes.id }$ \Comment{The output will be the name of the satellites and their unique database ID}
\State $\textbf{FROM } \textit{attributes }$ \Comment{\textit{attributes}: main table of the database}
\State $ \textbf{INNER JOIN } \textit{classification }$ Comment{\textit{classification}: table with information about type and mission objectives}
\State $ \textbf{ON } \textit{classification.environmental_monitoring = environmental_monitoring} $
\State $ \textbf{WHERE:} $
\State \hspace*{\algorithmicindent} $\textit{environmental_monitoring = } \text{ } \textcolor{purple}{objective} \textbf{ AND} $
\State \hspace*{\algorithmicindent} $\textit{resolution = } \text{ } \textcolor{purple}{meters}$
\EndProcedure
\end{algorithmic}
\hspace*{\algorithmicindent} \textbf{Output:}\\ \hspace*{\algorithmicindent} \textit{[NOVASAR-1, CARTOSAT-3, HUANJING-2B, HUANJING-2A, JILIN-01 GAOFEN-3J, ... (9 more JILIN-01 GAOFEN satellites), SENTINEL-1B, SENTINEL-1A]}
\end{algorithm}

From the second experiment, the user found that there are 16 satellites that have the desired properties, which can be also seen in the section of the \textit{Output} in the algorithm \ref{Request2}. An extension of this request is that the user wants to know the revisit time of a specific latitude from those satellites. For this example, the latitude of 48$^{\circ}$ was selected, which is the latitude of Munich. Therefore, by inserting into the software the ID of those satellites that were just found and the 48$^{\circ}$ latitude, the revisit time at this latitude is calculated for all 16 satellites. In the Figure \ref{revisit_time_at_latitude_48}, it can be seen that NOVASAR-1 is the satellite, which has the most frequent revisiting rate at Munich's latitude, which is less than 2 days.

\begin{figure}
\centering
\includegraphics[width=0.9\textwidth]{Images/revisit_time_at_latitude_48.png}
\caption{The revisit time at 48$^{\circ}$ latitude for a group of satellites, which group was found from a user's request (Algorithm \ref{Request2}).}
\label{revisit_time_at_latitude_48}
\end{figure}
%First find the id_list from the user.py. Then, for a different latitude and to find which satellite has the minimum revisit time use: main_code_db_latitude.py.

\bigskip
\section{Revisit time calculation of group of satellites/ \\constellation}
\bigskip

The revisit time calculation of a group of satellites or of a constellation is an additional capability of the software. A step in the procedure, which needs attention is the calculation of the mean anomaly of every satellite for the beginning of the simulation. This should be based on the same epoch. Since the TLE set that is stored for every satellite in the database is referred to a specific and probably different epoch than the rest of the satellites, a common epoch, which will serve as the starting point in time for all those satellites is needed.

Before presenting the results of a calculation of the revisit time of a satellite group, the groundtrack of the Sentinel-2A and Sentinel-2B for one orbit is shown, when their simulation starts at the same epoch. These two figures (\ref{groundtrack_Sentinel-2A}, \ref{groundtrack_Sentinel-2B}) are presented in order to be validated in Chapter \ref{chap:5}. 

\begin{figure}
\centering
\includegraphics[width=0.9\textwidth]{Images/groundtrack_Sentinel-2A.png}
\caption{The groundtrack of Sentinel-2A for one orbit. The starting point of the satellite's simulation is on the 28th of November 2020 at 00:00}
\label{groundtrack_Sentinel-2A}
\end{figure}

\begin{figure}
\centering
\includegraphics[width=0.9\textwidth]{Images/groundtrack_Sentinel-2B.png}
\caption{The groundtrack of Sentinel-2B for one orbit. The starting point of the satellite's simulation is on the 28th of November 2020 at 00:00.}
\label{groundtrack_Sentinel-2B}
\end{figure}

% Code: main_code_db_aggregated.py

\bigskip
The revisit time calculation of the Sentinel-2 constellation is presented in Figure \ref{revisit_time_of_SENTINEL-2A_SENTINEL-2B}. The average revisit time at the equator is around 5 days. Moreover, the revisit time calculation of the Sentinel-1 constellation was also computed. As shown in Figure \ref{revisit_time_of_SENTINEL-1A_SENTINEL-1B}, the average revisit time at the equator for this constellation is approximately 3 days.

\begin{figure}
\centering
\includegraphics[width=0.9\textwidth]{Images/revisit_time_of_SENTINEL-2A_SENTINEL-2B.png}
\caption{Average revisit time per longitude cell along the equator of the Sentinel-2 constellation. The length of the longitude cell was determined based on the swath width of the satellite's sensor, which in this case is 290 km. The average revisit time at the equator of Sentinel-2A and Sentinel-2B is 5.16 days.}
\label{revisit_time_of_SENTINEL-2A_SENTINEL-2B}
\end{figure}

\begin{figure}
\centering
\includegraphics[width=0.9\textwidth]{Images/revisit_time_of_SENTINEL-1A_SENTINEL-1B.png}
\caption{Average revisit time per longitude cell along the equator of the Sentinel-1 constellation. The length of the longitude cell was determined based on the swath width of the Interferometric Wide Swath Mode, which is 250 km. The average revisit time at the equator of Sentinel-1A and Sentinel-1B is 3.08 days.}
\label{revisit_time_of_SENTINEL-1A_SENTINEL-1B}
\end{figure}

Furthermore, the user can also find out what is the number of the crossing times of the satellite above the Earth's cells. As an example, the Sentinel-1 constellation for a simulation of 20 orbits is presented in Figure \ref{SENTINEL-1A_SENTINEL-1B_crossing_times.png} for showing this application's feature.

\begin{figure}
\centering
\includegraphics[width=0.9\textwidth]{Images/SENTINEL-1A_SENTINEL-1B_crossing_times.png}
\caption{The number of crossing times of the Sentinel-1 constellation above the Earth's cells. The creation of the cells was determined based on the swath width of the satellite's sensor.}
\label{SENTINEL-1A_SENTINEL-1B_crossing_times.png}
\end{figure}
% Run main_code_db_aggregated.py for Sentinel2A,B. BE CAREFUL in code you should have: type_of_sensor = 'active' !! This is because if the satellite is passive then, the groundtrack will be depicted only in the descending passes and it will be half!

% (If I want to create an aggregated revisit time of satellites, which have different swath widths, then a safe solution would be to use the maximum swath width of all of them and use it globally - since we need to have the same - there is one table with specific dimensions calculating the revisit time- and thus cells with certain grid interval. If I use the minimum swath width of the swath widths that the group of satellites have, then it might happen that some of the satellites do not produce an outcome of revisit time in this specific -and smaller than theirs- swath width. So, to be on the safe side - use the max swath width among all).

\bigskip
\section{Investigation of orbital characteristics of new satellite for a more frequent revisit time}
\bigskip

A user has also the possibility to find the position of a new satellite, which can achieve together with a group of other satellites a more frequent revisit time. As a first experiment, the Sentinel-2 constellation was used. Then, a prospective new satellite with similar orbital characteristics to the Sentinel-2 satellites was used to calculate the overall revisit time. The parameter that is varying is the mean anomaly ($M$), which means that the position of the prospective satellite in the orbital plane is changing throughout the different cases. The revisit time at the equator was then found for every case as it can be seen in Figure \ref{revisit_time_ofdoubleaxis_SENTINEL-2A_SENTINEL-2B_Example-satellite}. The two vertical green lines indicate the position of the Sentinel-2A and Sentinel-2B  in the orbital plane. Additionally, the reduction of the revisit time that the addition of the Example-satellite has brought is shown in the right side y-axis.

% Take Sentinel-2A and 2B. Think of a similar satellite with similar altitude, that its position varies (true anomaly varies - at the beginning ..? probably...) --> the position along the orbit. What is the position of the new satellite that has the maximum (the most often) revisit time?

\begin{figure}
\centering
\includegraphics[width=0.9\textwidth]{Images/revisit_time_ofdoubleaxis_SENTINEL-2A_SENTINEL-2B_Example-satellite.png}
\caption{Influence of the new satellite's mean anomaly change in the combined revisit time at equator of the Sentinel-2 constellation and this new Example-satellite. The green lines indicate the position of the Sentinel-2A and 2B satellites in the orbital plane.}
\label{revisit_time_ofdoubleaxis_SENTINEL-2A_SENTINEL-2B_Example-satellite}
\end{figure}

% The Sentinel-2A has mean anomaly: 93.2879064 degrees
% The Sentinel-2B has actually: 281.5507975, but in the figure we show mean anomaly: 275.5507975
% The Example-satellite's mean anomaly is varying. However when we plug in the database 0degrees, this is translated as 180.786359degrees, which we plot in this figure as 180degrees.

% Code: main_code_db_aggregated_experiment_Sent2.py
% The differences of this code with "code_db_aggregated.py" are: 1. Big "for loop" at the beginning with change_omega (BE CAREFUL TO BE CORRECT)., 2. In the determination of omega (line 204 or line 207) change it! If id == 249: (Here is the id of the Example_satellite), 3. I have set the list "multiple_average_aggregated_zerolat_without0", 4. I append the revisit time there (line 1072), 5. Change the "argument_of_perigee" parameter to have the correct amount of numbers + Make sure that all the id (satellites) that I use in the code are inside the database in the VE!!! If not update the database in VE.

A similar procedure of finding the optimal altitude of a new satellite was also implemented. The Sentinel-2 constellation was also part of this experiment and as it was calculated, its average altitude is around 790 km. Then, the prospective satellite with similar orbital characteristics was created in order to calculate the overall revisit time when its altitude is varying. This new satellite was also assumed to be in a sun-synchronous orbit as the Sentinel-2 constellation. The range of the different altitudes that were tested were 20 km below and above the average altitude. This range was found based on the fact that the period of the satellite is not changed substantially and thus in accordance with the inclination, the orbit is maintained a sun-synchronous one.

The experiments of finding the overall revisit time with the altitude change of this Example's satellite were performed for three different positions of the Example satellite in the orbital plane. Those positions, or else the values of the mean anomaly of the Example satellite for every case, were selected from the previous Figure in order to be ensured that the position of the third satellite will not be co-located with one of the two existing ones. For this reason, as can be seen in Figure \ref{revisit_time_ofdoubleaxis_SENTINEL-2A_SENTINEL-2B_Example-satellite}, the mean anomaly of the Sentinel-2A and 2B for the starting epoch of the simulation is approximately 93 and 273 degrees respectively. As a result, the three example cases that are presented, show the overall revisit time with the altitude change of the Example satellite when its mean anomaly is 50, 180 and 330 degrees (Figures \ref{revisit_time_ofdoubleaxisaltitude_SENTINEL-2A_SENTINEL-2B_Example-satellite_50deg}, \ref{revisit_time_ofdoubleaxisaltitude_SENTINEL-2A_SENTINEL-2B_Example-satellite_180deg}, \ref{revisit_time_ofdoubleaxisaltitude_SENTINEL-2A_SENTINEL-2B_Example-satellite_330deg}).

% I assume that the orbit is maintained as SSO (as I said).
% Source no1: wikipedia page of SSO. I used the equation of cos(i)=... by changing the semimajor axis according to the change of altitude. However, an "acceptable/ official" source to mention is Vallado4thedit p.890, Equation: 11-14
% Then, based on the graph that is in the pdf "2019_Joint_Agency_Commercial_Imagery_Evaluation" in p.15, I checked whether the value of the inclination is not on the line. This graph shows the relationship between inclination and time period in order the orbit to be maintained SSO.

\begin{figure}
\centering
\includegraphics[width=0.9\textwidth]{Images/revisit_time_ofdoubleaxisaltitude_SENTINEL-2A_SENTINEL-2B_Example-satellite_50deg.png}
\caption{Influence of the new satellite's altitude change in the combined revisit time at equator of the Sentinel-2 constellation and the Example-satellite. The yellow line indicates the altitude of the satellites of the Sentinel-2 constellation. \underline{The mean anomaly of the Example satellite is 50 degrees.}}
\label{revisit_time_ofdoubleaxisaltitude_SENTINEL-2A_SENTINEL-2B_Example-satellite_50deg}
\end{figure}

\begin{figure}
\centering
\includegraphics[width=0.9\textwidth]{Images/revisit_time_ofdoubleaxisaltitude_SENTINEL-2A_SENTINEL-2B_Example-satellite_180deg.png}
\caption{Influence of the new satellite's altitude change in the combined revisit time at equator of the Sentinel-2 constellation and the Example-satellite. The yellow line indicates the altitude of the satellites of the Sentinel-2 constellation. \underline{The mean anomaly of the Example satellite is 180 degrees.}}
\label{revisit_time_ofdoubleaxisaltitude_SENTINEL-2A_SENTINEL-2B_Example-satellite_180deg}
\end{figure}

\begin{figure}
\centering
\includegraphics[width=0.9\textwidth]{Images/revisit_time_ofdoubleaxisaltitude_SENTINEL-2A_SENTINEL-2B_Example-satellite_330deg.png}
\caption{Influence of the new satellite's altitude change in the combined revisit time at equator of the Sentinel-2 constellation and the Example-satellite. The yellow line indicates the altitude of the satellites of the Sentinel-2 constellation. \underline{The mean anomaly of the Example satellite is 330 degrees.}}
\label{revisit_time_ofdoubleaxisaltitude_SENTINEL-2A_SENTINEL-2B_Example-satellite_330deg}
\end{figure}

% Code: "main_code_db_aggregated_altitude.py"
% The range of altitudes that I used for the experiment with Sentinel-2: 770-810km and omega= 83 deg.
% The range of altitudes that I used for the experiment with Flocks: 465-505km.

The constellation of the Flock-3P satellites was also used for implementing the above mentioned experiment. Firstly, an Example satellite with similar orbital characteristics to the mean characteristics of the 37 Flock-3P satellites was used to calculate the overall revisit time. Since this group of satellites is nicely distributed in the orbital plane, the experiment that is presented, pertains to the different revisit times that are achieved due to the new satellite's altitude change. These results are depicted in Figure \ref{revisit_time_ofdoubleaxisaltitude_Flocks}, in which the mean altitude of 37 Flock-3P satellites is also indicated with the yellow line.


\begin{figure}
\centering
\includegraphics[width=0.9\textwidth]{Images/revisit_time_ofdoubleaxisaltitude_Flocks.png}
\caption{Influence of Example-satellite's altitude change in the combined revisit time at equator together with 37 Flock-3P satellites. The yellow line indicates the mean altitude of the satellites of the 37 Flock-3P satellites.}
\label{revisit_time_ofdoubleaxisaltitude_Flocks}
\end{figure}

\bigskip
\section{Investigation of revisit time of the entire Flock constellation}
\bigskip

The existence of proliferated constellations in the EO field has already been mentioned in table \ref{table:EO} and further information can also be found in table \ref{table:1} regarding the orbiting and operational satellites as of December 2020. However, the announcements of new launches, which will be part of already proliferated constellations, promise that the new larger fleet of satellites manages to improve the revisit time in all the part of the globe. One example company, which actualized a similar announcement recently is \textit{Planet}, promising that its constellations can revisit any part of the world seven to 12 times per day or else to revisit any place every two to four hours \footnote{\label{Flock_source}\textit{Available at: https://spacenews.com/planet-explore-2020/ (Accessed November 20, 2020)}}.

For this reason, an experiment, which can serve as a test of the revisit time of the satellite was performed. In this case, the entire Flock constellation was part of it namely 155 satellites, which are operational and in orbit as of December 2020. The average revisit time per longitude cell along the equator for this constellation can be found in Figure \ref{revisit_time_hours_of_155_Flocks}. As it can be seen, the average revisit time at equator for the constellation is almost 4 hours, which means that all the other places that have a non-zero latitude will have nearly the same or even more frequent revisit time, since the entire constellation is in near-polar orbits. 

\begin{figure}[!htb]
\centering
\includegraphics[width=0.9\textwidth]{Images/revisit_time_hours_of_155_Flocks.png}
\caption{Average revisit time along the equator of the entire Flock constellation - 155 satellites.}
\label{revisit_time_hours_of_155_Flocks}
\end{figure}

% Code: main_code_db_aggregated.py. I ran for all the Flocks, which are the presented all the presented id= 8-13 (and id=13!), 35-75, 77, 79-83, 86, 87, 96-142, 145-149, 151-160, 163-167, 170,171, 251-254, 256, 258-265
% BE CAREFUL!!! DONT PUT satellite_name_list in the title, or the name of the file!!!!
% Synolika, trexw gia ~ 255 Flock satellites (etsi lew kai ston pinaka sto Appendix and ston pinaka sto Chapter 1)



%\bigskip
%\section{Investigation of correlation between revisit time \& altitude}
%\bigskip
%
%\begin{figure}[!htb]
%\centering
%\includegraphics[width=0.9\textwidth]{Images/revisit_time_vs_altitude.png}
%\caption{Investigation of correlation between revisit time \& altitude} % Write caption smaller to wider swath width 
%\label{revisit_time_vs_altitude}
%\end{figure}
%
%% Code: revisit_altitude_plot.py
\chapter{Evaluation \& Discussion}
\lhead{Chapter 5 \emph{Evaluation}}
\label{chap:5}
%\autoref{cha:5}

\section{}
\bigskip

%Endexomenws na min xreiazetai ksexwristo kefalaio...

%The discussion will consist of argumentation. In other words, you investigate a phenomenon from several different perspectives. To discuss means to question your findings, and to consider different interpretations. Here are a few examples of formulations that signal argumentation:
%
%On the one hand … and on the other …
%However …
%… it could also be argued that …
%… another possible explanation may be …

\newpage
%\chapter*{References}
%\lhead{\emph{References}}
%\label{ref}



\chapter*{List of acronyms}
\addcontentsline{toc}{chapter}{List of acronyms}
\begin{abbrv}
% American Institute of Aeronautics and Astronautics (AIAA)
\item[\textit{ASI}] Agenzia Spaziale Italiana - Italian Space Agency
\item[\textit{CONAE}] Comisión Nacional de Actividades Espaciales of Argentina - \newline Argentinian Space Agency
\item[\textit{DMF}] Debris Mitigation Framework
\item[\textit{ECSS}] European Cooperation for Space Standardization
\item[\textit{EO}] Earth Observation
\item[\textit{EU}] European Union
\item[\textit{GEO}] Geostationary Orbit / Geosynchronous Equatorial Orbit
\item[\textit{GEOSS}] Global Earth Observation System of Systems
\item[\textit{GLAS}] Geoscience Laser Altimeter System
\item[\textit{HRMS}] High Resolution Multi-Spectral
\item[\textit{HSIS}] Hyper Spectral Image Simulator
\item[\textit{IADC}] Inter-Agency Space Debris Coordination Committee
\item[\textit{ISO}] International Organization for Standardization
\item[\textit{ISS}] International Space Station
\item[\textit{JACIE}] Joint Agency Commercial Imagery Evaluation
\item[\textit{JD}] Julian Date
\item[\textit{LE}] London Economics
\item[\textit{LEO}] Low Earth Orbit
\item[\textit{MELVO}] Moderately Elliptical Very Low Orbit
\item[\textit{MLST}] Mean Local Solar Time
\item[\textit{MSI}] Multi-Spectral Imager
\item[\textit{MSS}] Multi-Spectral Imaging System
\item[\textit{MSU}] Multi-Spectral Scanner Unit
\item[\textit{MUXCam}] Multi-Spectral Camera
\item[\textit{NIR}] Near Infrared
\item[\textit{ODCWG}] Orbital Debris Co-ordination Working Group
\item[\textit{OECD}] Organization for Economic Co-operation and Development
\item[\textit{PAN}] Panchromatic Imager
\item[\textit{SAR}] Synthetic Aperture Radar
\item[\textit{SIRAL}] SAR Interferometry Radar Altimeter
\item[\textit{SSO}] Sun-Synchronous Orbit
\item[\textit{SSR}] Space Sustainability Rating
\item[\textit{SWIR}] Short-Wave Infrared Light
\item[\textit{UN}] United Nations
\item[\textit{UNCOPUOUS}] United Nations Committee on the Peaceful Uses of Outer Space
\item[\textit{UNOOSA}] United Nations Office for Outer Space Affairs
\item[\textit{VNIR}] Visible and Near-infrared
\item[\textit{WEF}] World Economic Forum
\item[\textit{WFI}] Wide Field Imager
\item[\textit{WPM}] Wide Swath Panchromatic \& Multispectral Camera
\end{abbrv}


\begin{thebibliography}{100}
\bigskip

\bibitem{Cryo2ice_news} Amos J (2020) ESA and NASA line up satellites to measure Antarctic sea-ice. BBC News. Available at: https://www.bbc.com/news/science-environment-53326490 (Accessed July 30, 2020)
\bibitem{Cosmos} Battagliere ML, Daraio MG, Lenti F, Pisani AR, Coletta A (2018) Cosmo-Skymed and the ASI-CONAE cooperation: The SIASGE Programme. 68th International Astronautical Congress (IAC), Bremen, Germany, 1-5 October 2018.
\bibitem{Belward 2015} Belward AS, Skøien JO (2015) Who launched what, when and why; trends in global land-cover observation capacity from civilian Earth observation satellites. ISPRS Journal of Photogrammetry and Remote Sensing, May; 103: 115–128.
\bibitem{Blake} Blake JA, Chote P, Pollacco D, Feline W, Privett G, Ash A, Eves S, Greenwood N, Marsh TR, Veras D, Watson C (2020) DebrisWatch I: A survey of faint geosynchronous debris. Advances in Space Research
\bibitem{Amazon} Boyle A (April 4, 2019) Amazon to offer broadband access from orbit with 3,236-satellite ‘Project Kuiper’ constellation. GeekWire. Available at: https://www.geekwire.com/2019/amazon-project-kuiper-broadband-satellite/ (Accessed July 1, 2020)
\bibitem{Braun} Braun V, Lemmens S (2020) Addressing space debris mitigation in satellite mission design. 71st International Astronautical Congress (IAC) – The CyberSpace Edition, 12-14 October 2020
\bibitem{Cadman} Cadman J (May 18, 2020) Op-ed | An unexpected effect: the industry’s recent challenges prove the importance of space. Space News. Available at: https://spacenews.com/op-ed-an-unexpected-effect-the-industrys-recent-challenges-prove-the-importance-of-space/ (Accessed July 1, 2020)
\bibitem{Campbell} Campbell BA, McCandless PW (1996) Introduction to space sciences and spacecraft applications. Elsevier
\bibitem{Committee 2019} Committee on the Peaceful Uses of Outer Space (2019) The “Space2030” Agenda: Space as a driver of sustainable development, Vienna, Report No.: 28th.
\bibitem{Crisp 2020} Crisp NH, Roberts PC, Livadiotti S, Oiko VTA, Edmondson S, Haigh SJ, Huyton C, Sinpetru LA, Smith KL, Worrall SD, Becedas J (2020) The benefits of very low Earth orbit for Earth observation missions. Progress in Aerospace Sciences, 117
\bibitem{Christopherson} Christopherson JB, Chandra SNR, Quanbeck JQ (2019) 2019 Joint Agency Commercial Imagery Evaluation — Land remote sensing satellite compendium (No. 1455). US Geological Survey
\bibitem{Anti-satellite} David L (Feb 2, 2007) China's Anti-Satellite Test: Worrisome Debris Cloud Circles Earth. SPACE.com [Internet] Available at: https://www.space.com/3415-china-anti-satellite-test-worrisome-debris-cloud-circles-earth.html (Accessed July 1, 2020)
\bibitem{Elachi} Elachi C, Van Zyl JJ (2006) Introduction to the physics and techniques of remote sensing (Vol. 28). John Wiley \& Sons
\bibitem{Erwin} Erwin S (May 18, 2020) SpaceX rideshare program putting downward pressure on prices [Internet]. Space News. Available at: https://spacenews.com/spacex-rideshare-program-putting-downward-pressure-on-prices/ (Accessed July 1, 2020)
\bibitem{cooperation} ESA (2020) New space satellite pinpoints industrial methane emissions. Available at: https://www.esa.int/Applications/Observing_the_Earth/Copernicus/Sentinel-5P/New_Space_satellite_pinpoints_industrial_methane_emissions (Accessed October 30, 2020)
\bibitem{ESA EO} ESA Space Solutions (August 11, 2020) Newcomers Earth Observation Guide. Report Version 1.9 Available at: https://business.esa.int/newcomers-earth-observation-guide (Accessed November 1, 2020)
\bibitem{ESA 2020} ESA Space Debris Office - ESOC, (Sep 2020) ESA’s Annual Space Environment Report [Internet]. Report No.: 4.0, p. 88. Available at: https://www.sdo.esoc.esa.int/environment_report/Space_Environment_Report_latest.pdf
%Statistics: https://sdup.esoc.esa.int/discosweb/statistics/
\bibitem{LE_Esteve} Esteve R (2020) Earth Observation: a tool for a more resilient and sustainable world? London Economics: Space In Focus, Issue 2
\bibitem{Space sustainability} Foust J (May 7, 2019) Consortium to develop “space sustainability” rating system. Space News. Available at: https://spacenews.com/consortium-to-develop-space-sustainability-rating-system/ (Accessed July 1, 2020)
\bibitem{Griffin} Griffin M (2010) Orbit/Spectrum Allocation Procedures. Space Services Department ITU Radiocommunication Bureau (BR) Conference of ITU in Bangkok, Thailand
\bibitem{Hallex} Hallex MA, Cottom TS (2020) Proliferated Commercial Satellite Constellations - Implications for National Security. Joint Force Quarterly (JFQ) 97
\bibitem{Kepler} Henry C (August 29, 2018) Kepler Communications opens launch bids for Gen-1 LEO constellation. Space News. Available at: https://spacenews.com/kepler-communications-opens-launch-bids-for-gen-1-leo-constellation/ (Accessed July 2, 2020)
\bibitem{Viasat} Henry C (April 24, 2020) Viasat gets FCC approval for MEO constellation. Space News. Available at: https://spacenews.com/viasat-gets-fcc-approval-for-meo-constellation/ (Accessed July 1, 2020)
\bibitem{IADC 2007} Inter-Agency Space Debris Coordinate Committee (IADC) -  Steering Group and Working Group 4 (2007) IADC Space Debris Mitigation Guidelines. Report No.: 22.4. Revision 1
%Available at: https://www.unoosa.org/documents/pdf/spacelaw/sd/IADC-2002-01-IADC-Space_Debris-Guidelines-Revision1.pdf
\bibitem{Kelso 2009} Kelso TS (2009) Analysis and Implications of the Iridium 33-Cosmos 2251 Collision. Advanced Maui Optical and Space Surveillance Technologies Conference
\bibitem{Klinkrad 2006} Klinkrad H (2006) Space Debris: Models and Risk Analysis. Springer Verlag, 1st ed., p.430
\bibitem{Klinkrad 2009} Klinkrad H, Johnson NL (2009) Space debris environment remediation concepts. 5th European Conference on Space Debris. In Darmstadt, Germany; (ESA SP-672). 
\bibitem{Kramer 2002} Kramer H (2002) Observation of the Earth and its Environment. Springer Verlag, 4th ed., p.1514, Updated version: Apr, 2020 Available at: https://directory.eoportal.org/web/eoportal/kramer
\bibitem{Newspace} Kulu E (2020) NewSpace Index - Overview of commercial satellite constellations. [Database] Available at: https://www.newspace.im/
\bibitem{Lapaine} Lapaine M, Usery EL (2017) Choosing a map projection. Cham: Springer
\bibitem{Letizia 2019} Letizia F, Lemmens S, Bastida Virgili B, Krag H. (2019) Application of a debris index for global evaluation of mitigation strategies. Acta Astronautica. 161:348–62.
\bibitem{Luo} Luo X, Wang M, Dai G, Chen X (2017) A novel technique to compute the revisit time of satellites and its application in remote sensing satellite optimization design. International Journal of Aerospace Engineering
\bibitem{Oneweb_bankruptcy} Malik T (March 28, 2020) OneWeb, a satellite internet startup, files for Chapter 11 bankruptcy. SPACE.com [Internet] Available at: https://www.space.com/oneweb-satellite-internet-startup-files-for-bankruptcy.html (Accessed July 1, 2020)
\bibitem{Meseguer} Meseguer J, Pérez-Grande I, Sanz-Andrés A (2012) Keplerian orbits - Spacecraft thermal control. Elsevier p.39-57
\bibitem{Satellogic} Mohney D (January 16, 2019) Satellogic signs launch contracts for 90 imaging satellites. Space It Bridge. Available at: https://www.spaceitbridge.com/satellogic-signs-launch-contracts-for-90-imaging-satellites.htm (Accessed July 1, 2020)
\bibitem{Montenbruck} Montenbruck O, Gill E, Lutze F (2002) Satellite orbits: models, methods, and applications. Appl. Mech. Rev., 55(2), B27-B28
% Source in the pc: ESPACE-TUM > 1st semester > Orbit Mechanics
\bibitem{muranaka} Muranaka T, Okumura T, Ueno K, Otsuka S, Ohkawa Y (2020) Space debris removal using drag force intensifier applying charged membrane: System design and feasibility studies. 71st International Astronautical Congress (IAC) – The CyberSpace Edition, 12-14 October 2020. 
%\bibitem{NASA} National Aeronautics and Space Administration (NASA) (2011) USA Space Debris Environment, Operations, and Policy Updates [Internet]. 48th Session of the Scientific and Technical Subcommittee Committee on the Peaceful Uses of Outer Space United Nations. Available from: https://www.unoosa.org/pdf/pres/stsc2011/tech-31.pdf
\bibitem{UNOOSA} Office for Outer Space Affairs (2020) Access to Space for All. Symposium in Vienna International Center - February 2020. Available at: https://www.unoosa.org/oosa/en/ourwork/access2space4all/index.html (Accessed July 30, 2020)
\bibitem{Cryo2ice} Ramsayer K (2020) Syncing NASA Laser, ESA Radar for a New Look at Sea Ice. NASA's Goddard Space Flight Center. Available at: https://www.nasa.gov/feature/goddard/2020/syncing-nasa-laser-esa-radar-for-a-new-look-at-sea-ice (Accessed July 30, 2020)
\bibitem{Value UK} Sadlier G, Flytkjær R, Sabri F, Robin N. (2018) Value of satellite-derived Earth Observation capabilities to the UK Government today and by 2020 - Evidence from nine domestic civil use cases. UK: London Economics (LE).
\bibitem{active} Shan M, Guo J, Gill E (2016) Review and comparison of active space debris capturing and removal methods. Progress in Aerospace Sciences, 80, 18-32.
\bibitem{CNBC} Sheetz M (Jun 9, 2020) How Planet’s new satellite fleet will bring detailed images of places on Earth up to 12 times a day. Investing in Space - CNBC. Available at: https://www.cnbc.com/2020/06/09/planets-skysats-will-take-images-up-to-12-times-a-day-launched-with-help-of-spacex.html (Accessed July 1, 2020)
\bibitem{Somma 2019} Somma GL, Lewis HG, Colombo C (2019) Sensitivity analysis of launch activities in Low Earth Orbit. Acta Astronautica, 158:129–39.
\bibitem{BRICS} Spacewatch Africa (2020) BRICS member states negotiating Earth Observation Satellite sharing framework. Business Intelligence About Space Activities. Available at: https://spacewatch.global/2019/12/brics-member-states-negotiating-earth-observation-satellite-sharing-framework/ (Accessed July 17, 2020)
\bibitem{takeichi} Takeichi N, Tachibana N (2020) A space debris removal strategy using a collision with small relative velocity. 71st International Astronautical Congress (IAC) – The CyberSpace Edition, 12-14 October 2020
\bibitem{pLEO} Taverney T (March 5, 2020) Op-ed | Proliferated LEO is risky but necessary. Space News. Available at: https://spacenews.com/op-ed-proliferated-leo-is-risky-but-necessary/ (Accessed July 1, 2020)
\bibitem{UCS} Union of Concerned Scientists (2020) UCS Satellite Database [Internet]. Published: Dec 8, 2005. Updated: August 1, 2020. Available at: https://www.ucsusa.org/resources/satellite-database (Accessed November 1, 2020)
\bibitem{Undseth} Undseth M, Jolly C, Olivari M (2020) Space sustainability: The economics of space debris in perspective. OECD Science, Technology and Industry Policy Papers, No. 87, OECD Publishing, Paris
\bibitem{UNOOSA} United Nations Office for Outer Space Affairs (2010) Space Debris Mitigation Guidelines of the Committee on the Peaceful Uses of Outer Space. Vienna; Available at: https://www.unoosa.org/pdf/publications/st_space_49E.pdf (Accessed July 30, 2020)
\bibitem{Vallado} Vallado DA (2001) Fundamentals of astrodynamics and applications (Vol. 12) 4th edition. Springer Science \& Business Media
\bibitem{TLE_Vallado} Vallado DA, Cefola PJ (2012) Two-line element sets–practice and use. In 63rd International Astronautical Congress, Naples, Italy
\bibitem{Wang} Wang P (2013) Tragedy of Commons in Outer Space-The Case of Space Debris. In 64th International Astronautical Congress
\bibitem{crowded} Wattles J (Oct 9, 2020) Space is becoming too crowded, Rocket Lab CEO warns. CNN Business. Available at: https://edition.cnn.com/2020/10/07/business/rocket-lab-debris-launch-traffic-scn/index.html
\bibitem{NRO} Werner D (Jun 21, 2020) NRO to award multiple imagery contracts by year’s end. Space News. Available at: https://spacenews.com/nro-to-award-multiple-imagery-contracts-by-years-end/ (Accessed July 30, 2020)
\bibitem{Hongyun} Zhao Lei (Dec 2018) China begins space-based broadband project. Jiuquan Satellite Launch Center | chinadaily.com.cn Available at: https://www.chinadaily.com.cn/a/201812/22/WS5c1d82d6a3107d4c3a002337.html (Accessed July 1, 2020)
%Lemmens, S., & Krag, H. (2014). Two-Line-Elements-Based Maneuver Detection Methods for Satellites in Low Earth Orbit. Journal of Guidance, Control, and Dynamics, 37(3), 860–868.



%The easiest thing in Zotero --> Ctrl + A, right-click, Create bibliography from items, Copy to Clipboard and then, copy paste to all of the \bibitem and the {...} for the citation.

\end{thebibliography}


\newpage
\chapter*{Appendix A}
\lhead{\emph{Appendix A}}
\label{app}
\addcontentsline{toc}{chapter}{Appendix A}



\begin{center}
\captionof{table}{The largest constellations and well-known EO satellites in LEO region with their characteristics as of December 2020. These data together with the orbital parameters of each satellite are used for the application. They are stored in a PostgreSQL database. (Part A)}
\vspace{3mm}
\begin{adjustbox}{width=1.05\textwidth}
\begin{tabular}{||m{2.5cm}|m{1.9cm}|m{2cm}|m{1.5cm}|m{1.5cm}|m{1.5cm}|m{1.7cm}|m{2cm}||}
	\hline
	\textbf{\thead{Constellation/\\satellite\\(\# of satellites$^*$)}} & \textbf{\thead{Launch\\date}} & \textbf{\thead{Operational \\lifetime\\(years)}} & \textbf{\thead{Sensor}} & \textbf{\thead{Swath \\width (km)}} & \textbf{\thead{Resolution\\(m)}} & \textbf{\thead{Mission\\type $^{**}$}} & \textbf{\thead{Mission\\objective$^{***}$}}\\\hline \hline
	\thead{Alsat-2\\(2)} & \thead{2010 \& 2016} & \thead{5} & \thead{passive} &\thead{600} & \thead{1.5-6} & \thead{Civil} & \thead{Environment,\\Emergency}\\\hline
	\thead{Blacksky Global\\(4)} & \thead{2019 \& 2020} & \thead{3} & \thead{passive} &\thead{30} & \thead{0.9-1.1} & \thead{Commercial} & \thead{Emergency,\\Science}\\\hline
	\thead{Cartosat-2\&3\\(5)} & \thead{2016 - 2019} & \thead{5} & \thead{passive} &\thead{250} & \thead{0.65-2} & \thead{Civil} & \thead{Environment}\\\hline
	\thead{CBERS-4A\\(1)} & \thead{2019} & \thead{5} & \thead{passive} &\thead{400} & \thead{16} & \thead{Civil} & \thead{Environment}\\\hline
	\thead{Cryosat-2\\(1)} & \thead{2010} & \thead{3.5} & \thead{active} &\thead{100} & \thead{250} & \thead{Civil} & \thead{Environment,\\Science}\\\hline
	\thead{CSG-1\\(1)} & \thead{2019} & \thead{7} & \thead{active} &\thead{200} & \thead{0.8-6} & \thead{Civil} & \thead{Environment,\\Emergency,\\Science}\\\hline
	\thead{Flock-2K \&\\Flock-3M\&3P\\(65)} & \thead{2017} & \thead{3} & \thead{passive} &\thead{100} & \thead{3.7} & \thead{Commercial} & \thead{Environment,\\Emergency}\\\hline
	\thead{Flock-2P\\(6)} & \thead{2016} & \thead{3} & \thead{passive} &\thead{100} & \thead{3.5} & \thead{Commercial} & \thead{Environment}\\\hline
\end{tabular}
\label{table:1}
\end{adjustbox}
\end{center}
\footnotesize{$^*$ {\scriptsize Number of operating satellites as of December 2020.}}\\
\footnotesize{$^{**}$ {\scriptsize The first classification of an EO mission (Figure \ref{classification_1st_layer}).}}\\
\footnotesize{$^{***}$ {\scriptsize The main objective of an EO mission (Figure \ref{classification_2nd_layer}).}}


\pagebreak

\begin{center}
\captionof{table}{The largest constellations and well-known EO satellites in LEO region with their characteristics as of December 2020. (Part B)}
\vspace{3mm}
\begin{adjustbox}{width=1.05\textwidth}
\begin{tabular}{||m{2.5cm}|m{1.9cm}|m{2cm}|m{1.5cm}|m{1.5cm}|m{1.5cm}|m{1.7cm}|m{2cm}||}
	\hline
	\textbf{\thead{Constellation/\\satellite\\(\# of satellites$^*$)}} & \textbf{\thead{Launch\\date}} & \textbf{\thead{Operational \\lifetime\\(years)}} & \textbf{\thead{Sensor}} & \textbf{\thead{Swath \\width (km)}} & \textbf{\thead{Resolution\\(m)}} & \textbf{\thead{Mission\\type$^{**}$}} & \textbf{\thead{Mission\\objective$^{***}$}}\\\hline \hline
	\thead{Flock-3K \&\\Flock-3R\&3S\\(24)} & \thead{2018} & \thead{3} & \thead{passive} &\thead{100} & \thead{3.7} & \thead{Commercial} & \thead{Environment,\\Emergency}\\\hline	
	\thead{Flock-4A\&4P\\(26)} & \thead{2019} & \thead{3} & \thead{passive} &\thead{100} & \thead{3.7} & \thead{Commercial} & \thead{Environment,\\Emergency}\\\hline	
	\thead{Flock-4V\&4E\\(34)} & \thead{2020} & \thead{3} & \thead{passive} &\thead{100} & \thead{3.7} & \thead{Commercial} & \thead{Environment,\\Emergency}\\\hline
	\thead{GHGSAT-C\&D\\(2)} & \thead{2019 \& 2020} & \thead{3} & \thead{passive} &\thead{15} & \thead{25,50} & \thead{Commercial} & \thead{Environment}\\\hline
	\thead{GOMX-4\\(2)} & \thead{2018} & \thead{3} & \thead{passive} &\thead{200} & \thead{70} & \thead{Civil} & \thead{Environment,\\Science}\\\hline
	\thead{Huanjing-2\\(2)} & \thead{2020} & \thead{3} & \thead{passive} &\thead{700} & \thead{16,48} & \thead{Defence} & \thead{Environment,\\Emergency}\\\hline
	\thead{Icesat-2\\(1)} & \thead{2018} & \thead{5} & \thead{active} &\thead{100} & \thead{0.7} & \thead{Civil} & \thead{Environment,\\Science}\\\hline
	\thead{Jilin-01 Gaofen\\(12)} & \thead{2019 \& 2020} & \thead{8} & \thead{passive} &\thead{100} & \thead{0.75-3} & \thead{Commercial} & \thead{Environment,\\Emergency}\\\hline
	\thead{Kanopus-V\\(4)} & \thead{2018} & \thead{5} & \thead{passive} &\thead{250} & \thead{2.5-25} & \thead{Civil} & \thead{Environment,\\Emergency,\\Science}\\\hline
	\thead{Landmapper-BC\\(4)} & \thead{2017 \& 2018} & \thead{5} & \thead{passive} &\thead{220} & \thead{22} & \thead{Commercial} & \thead{Environment}\\\hline
	\thead{Novasar-1\\(1)} & \thead{2018} & \thead{7} & \thead{active} &\thead{200} & \thead{6,20,40} & \thead{Civil} & \thead{Environment,\\Emergency}\\\hline
	\thead{Nusat\\(18)} & \thead{2016 - 2020} & \thead{4} & \thead{passive} &\thead{150} & \thead{1} & \thead{Commercial} & \thead{Environment}\\\hline
\end{tabular}
\label{table:2}
\end{adjustbox}
\end{center}
\footnotesize{$^*$ {\scriptsize Number of operating satellites as of December 2020.}}\\
\footnotesize{$^{**}$ {\scriptsize The first classification of an EO mission (Figure \ref{classification_1st_layer}).}}\\
\footnotesize{$^{***}$ {\scriptsize The main objective of an EO mission (Figure \ref{classification_2nd_layer}).}}


\pagebreak

\begin{center}
\captionof{table}{The largest constellations and well-known EO satellites in LEO region with their characteristics as of December 2020. (Part C)}
\vspace{3mm}
\begin{adjustbox}{width=1.05\textwidth}
\begin{tabular}{||m{2.5cm}|m{1.9cm}|m{2cm}|m{1.5cm}|m{1.5cm}|m{1.5cm}|m{1.7cm}|m{2cm}||}
	\hline
	\textbf{\thead{Constellation/\\satellite\\(\# of satellites$^*$)}} & \textbf{\thead{Launch\\date}} & \textbf{\thead{Operational \\lifetime\\(years)}} & \textbf{\thead{Sensor}} & \textbf{\thead{Swath \\width (km)}} & \textbf{\thead{Resolution\\(m)}} & \textbf{\thead{Mission\\type$^{**}$}} & \textbf{\thead{Mission\\objective$^{***}$}}\\\hline \hline
	\thead{Perusat-1\\(1)} & \thead{2016} & \thead{10} & \thead{passive} &\thead{100} & \thead{0.7-2} & \thead{Civil} & \thead{Science,\\Environment,\\Emergency}\\\hline	
	\thead{Prisma\\(1)} & \thead{2019} & \thead{5} & \thead{passive} &\thead{100} & \thead{5-30} & \thead{Civil} & \thead{Environment,\\Science}\\\hline	
	\thead{Radarsat-2\\(1)} & \thead{2007} & \thead{7} & \thead{active} &\thead{500} & \thead{100} & \thead{Civil} & \thead{Environment,\\Security}\\\hline	
	\thead{RCM\\(3)} & \thead{2019} & \thead{7} & \thead{active} &\thead{500} & \thead{100} & \thead{Civil} & \thead{Environment,\\Emergency}\\\hline
	\thead{Saocom-1A\\(1)} & \thead{2018} & \thead{5} & \thead{active} &\thead{400} & \thead{100} & \thead{Civil} & \thead{Environment,\\Emergency}\\\hline
	\thead{Sentinel-1\\(2)} & \thead{2014 \& 2016} & \thead{7.25} & \thead{active} &\thead{250, 400} & \thead{5} & \thead{Civil} & \thead{Environment,\\Emergency,\\Science,\\Security}\\\hline
	\thead{Sentinel-2\\(2)} & \thead{2015 \& 2017} & \thead{7.25} & \thead{passive} &\thead{290} & \thead{10-20} & \thead{Civil} & \thead{Environment,\\Emergency,\\Science}\\\hline
	\thead{Sentinel-3\\(2)} & \thead{2016 \& 2018} & \thead{7} & \thead{active} &\thead{1270} & \thead{300} & \thead{Civil} & \thead{Environment,\\Science}\\\hline
	\thead{Sentinel-5P\\(1)} & \thead{2017} & \thead{7} & \thead{passive} &\thead{2600} & \thead{5500} & \thead{Civil} & \thead{Environment,\\Emergency,\\Science}\\\hline
	\thead{Sentinel-6\\(1)} & \thead{2020} & \thead{5.5} & \thead{active} &\thead{500} & \thead{-} & \thead{Civil} & \thead{Environment,\\Emergency,\\Science}\\\hline
	\thead{Skysat\\(19)} & \thead{2016 - 2020} & \thead{6} & \thead{passive} &\thead{200} & \thead{0.5-2} & \thead{Commercial} & \thead{Environment,\\Emergency}\\\hline
	\thead{Superview-1\\(2)} & \thead{2018} & \thead{8} & \thead{passive} &\thead{60} & \thead{0.5-2} & \thead{Commercial} & \thead{Environment}\\\hline
\end{tabular}
\label{table:3}
\end{adjustbox}
\end{center}
\footnotesize{$^*$ {\scriptsize Number of operating satellites as of December 2020.}}\\
\footnotesize{$^{**}$ {\scriptsize The first classification of an EO mission (Figure \ref{classification_1st_layer}).}}\\
\footnotesize{$^{***}$ {\scriptsize The main objective of an EO mission (Figure \ref{classification_2nd_layer}).}}

%\newpage
%\chapter*{Appendix B}
%\lhead{\emph{Appendix B}}
%\label{appB}
%\addcontentsline{toc}{chapter}{Appendix B}
% In general, the Python packages used in this work are: numpy, math, matplotlib, datetime, configparser, seaborn, pandas, cartopy.crs \& psycopg2.

\end{document}