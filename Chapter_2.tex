\chapter{Theoretical Framework}
\lhead{Chapter 2 \emph{Theoretical Framework}}
\label{chap:2}
%\autoref{chap:2}

\section{The problem of overpopulation}
\bigskip

Space is a non-infinite environment and inherently %=έμφυτα, εγγενώς
international. At the same time, as it is stated in the Outer Space Treaty by UNOOSA, there is no ownership in space. However, it seems that nowadays it can be linked to the phenomenon of Tragedy of Commons. Based on this concept written by the British economist Lloyd, the shared-resource system is spoiled by individual users who act based on their own benefit and against the common good. This notion describes the current situation of the space industry, which comes as an international problem.

%There is a big problem especially in LEO
The problem of overpopulation is mainly encountered in LEO. To be more exact, approximately 75\% of all cataloged objects are located in LEO, which results in several negative effects. Not only the risk of collision becomes bigger, but also the safety of human spaceflight is at risk, since ISS operated and performs in the LEO altitudes (~350-400km). \cite{Kramer 2002} Additional side effects are presented below.

%Reason why it happens/ Causes
The reasons behind the problem of overpopulation are manifold. Firstly, one root cause is the incremental number of launched objects, which is also highly connected to the free market, the commercial district and the free competition between the private corporations. %(no trading restrictions)
One argument to launch more satellites by the \textit{Planet} company is that the commercial market has opened up since the daily revisits have been increased with the newly launched satellites and the offered resolution is higher. As a direct result, there is more demand as the costumers show greater interest, which is cited as a good reason to launch more objects. \cite{CNBC}

Even though the risk of collision is one of the side effects of overpopulation, in the event of an actual impact, the particles being created will definitely deteriorate the situation. This is in fact the vicious circle predicted by NASA's scientist Kessler. (Chapter \ref{chap:1.1}) Thus, as a second reason of the problem of overpopulation, the accidental break-ups can be mentioned. The first accidental collision, which took place in 2009 at LEO, is undoubtedly a benchmark in space history. The number of debris that this collision between the US \textit{Iridium-33} and the Russian \textit{Kosmos-2252} left behind was more than 2,300 pieces with average size of 12.5cm and average weight 1.1kg. \cite{Kelso 2009} This impact is an indication that a collisional cascading might happen soon, if action is not taken.
%Between the US commercial non-operational Iridium-33 and the Russian military Kosmos-2251.

Another reason behind the overpopulation problem is the satellite interceptions by surface-launched missiles. The anti-satellite tests are weapons, which not only aim at destroying satellite, but also their action creates uncountable space debris. As an example, in 2007, the intentional destruction of the chinese \textit{FengYun-1C} satellite doubled the amount of debris at an altitude of about 800 km, leading to a 30\% increase in the total population of debris at that time. \cite{Anti-satellite} Last but not least, the fact that there is lack of space traffic regulations, and mitigation guidelines, which are legally binding under international law, such problem is difficult to be handled.

\bigskip

Some of the negative effects of this crucial issue has already been emerged as they are discussed below, whereas some others are looming. The number of launched objects and the number of space debris is proportional to the possibility of impact between two objects/ particles. In other words, in the case of overpopulation, the risk of collision is higher. More specifically, after the first accidental aforementioned collision in 2009, the remote sensing satellites \textit{ERS-2} and \textit{Envisat} due to their proximity to the region of impact, have increased their risk of a secondary catastrophic collision by almost a factor 2. \cite{Klinkrad 2009}

Not only the possibility of impact is increasing, but there are also economical costs involved. In this way, a potential risk of collision has led to find another ways of preventing clash between satellites. These are the creation of shields for protecting satellites from space debris, the need of surveillance and tracking of the orbiting objects, and the replacement of missions altogether to be equipped with the latest technology and to be able to maneuver if needed. Even though these measures had proved to be helpful, their operation do not solve the root of the problem.

Another possible aftermath of the problem is linked to the deterioration of socio-economic conditions. Important space services could be lost, such as weather forecasting, emergency management, as well as climate monitoring. Additionally, space-based communications and internet could be affected, resulting in a required disruption of the communication and transactions globally. \cite{Undseth}

\bigskip

Every orbital plane has a certain capacity, which should be taken into account when a future satellite constellation is going to be launched. In general the concept of orbital capacity was formulated in order to estimate the global evolution of the space environment. Towards this goal, the calculation of the Space Debris Index is essential. It is based on the collision risk of operational satellites and the collision effect, as well as the explosion probability and effect. This index can be applied to both single objects and to the whole environment. As a direct consequence, when there is more capacity, then the reliability, which is connected to the Post Mission Disposal success rate, is low. As it can also be seen in the Chapter \ref{chap:mitigation}, the concept of orbital capacity will be used to sharpen the current space debris implementation guidelines. \cite{Letizia 2019}


\bigskip
\section{Mitigation principles}
\label{chap:mitigation}
\bigskip

There are multiple ways of dealing with the problem of overpopulation. %confronting a problem
To begin with, the methods, which help towards the avoidance of collisions between the already launched satellites are the following. One approach is to track debris by using ground-based systems. Objects larger than 10cm in LEO and larger than 1m in GEO can be tracked. \cite{Kramer 2002} Smaller objects than the aforementioned sizes, not to mention objects in the mm scale, are difficult to be detected due to the sensitivity limits of the instruments, such as radars and telescopes.

Another way of dealing with the problem, which requires more effort, time and bears extra costs, is the method of collision avoidance messages and subsequently the action of maneuvering a satellite. In case there is an estimation of a possible collision with at least one of the involved objects being operational, maneuvering is the instinctive reaction of the satellite operators. In a similar manner, a method, which will have a great impact on reducing the risk of collision especially with the big, non-operational satellites that are still orbiting Earth, such as \textit{Envisat}, is the active space debris removal. Some of the projects working in this direction is the \textit{e.Deorbit} by ESA and the EPFL's \textit{CleanSpace One}.

Several ways of deorbiting can also help into mitigating overpopulation. The uncontrolled reentry is one of them, which occurs when the satellite is approximately at 600km and it is also referred as natural drag. This way of uncontrolled reentry is proportional to the ballistic coefficient, which is the effective drag area divided by the mass. Thus, in case of having a small satellite, the decay is longer as compared to the decay of a larger one. Although this method is a popular choice of dealing with the overpopulation, it still poses a threat, since the satellite passes unguided through lower altitudes. On the other hand, there is the controlled reentry, which is a deorbiting option when the satellite is located approximately at 800 to 1,000km. In this method, the use of thrusters or deployment of drag devices is one of the alternatives. Furthermore, when the satellite is in higher altitudes than of 1,000km, a considerate thought is to maneuver the satellite to one of the disposal regions above the 2,000km, in which the objects will not interfere with the future space operations. \cite{NASA}


\bigskip
%A method, which prevents the further acting of deorbiting after the end of life of a LEO satellite is its placement to a MELVOs orbit (Moderately Elliptical Very Low Orbits) instead of a LEO one. 
A convenient method, which has been prepared and programmed for the deorbit of the satellite without further action after the end of life due to the orbit selection, is the following. Instead of placing a future LEO satellite in a LEO orbit, it is placed in a MELVOs orbit (Moderately Elliptical Very Low Orbits). The characteristics of a MELVO orbit is that the perigee is <300km, the apogee <500km and the eccentricity is between 0.015 and 0.030. Some of the advantages of this method is that when the orbit maintenance stops, the debris population can decay in this altitude and reenter the atmosphere in a short time, due the elliptical shape of the orbit. As a reference, from a circular orbit of 300km, it reenters the atmosphere in about 23 days at solar maximum and 70 days at solar minimum. Also, if there is a failure in the orbit maintenance burns, the orbit will turn into a circular one without loss of delta V. Another positive aspect of this idea is that since the perigee has low altitude, there is a better resolution on Earth and also reduced orbital debris issues, both in terms of collision probability and contribution to the long term debris problem. \cite{Kramer 2002} On the other hand, the disadvantages of this concept are that delta V is required, the coverage is reduced compared to a LEO orbit and that it has a reduced design life. Nevertheless, it has potentially lower cost per year.

\bigskip
%RAISING CONSCIOUSNESS
There have been made many important steps towards raising the consciousness about the overpopulation problem, as well as the enactment of rules regarding the capacity and orbit allocation. The most significant milestones that showed an international cooperation at a technical level are the following.

In 1993 the \textit{Inter-Agency Space Debris Coordination Committee} (IADC) was founded. Its goal is to coordinate the efforts among the various space agencies that are involved in and structure a plan of dealing with the debris orbiting the Earth. Some years later, in 2002 IADC published for the first time a report about Space Debris Mitigation Guidelines. \cite{UNOOSA} %The space debris mitigation was performed in 2002 by IADC and presented at UNCOPUOUS in 2003.
These guidelines were presented at the \textit{United Nations Committee on the Peaceful Uses of Outer Space} (UNCOPUOUS), which is a committee established by \textit{United Nations Office for Outer Space Affairs} (UNOOSA). \cite{IADC 2007} In 2003 another group focused on similar goals was created – the \textit{Orbital Debris Co-ordination Working Group} (ODCWG). It was established by unanimous agreement of \textit{International Organization for Standardization} (ISO). \cite{Klinkrad 2006}
%The two main groups that work towards these goal are the UNCOPUOUS and the ODCWG.

The first five mitigation guidelines, which were presented at the UNCOPUOUS in 2003, are related to the following concepts: prevention of the release of a mission related object, implementation of collision avoidance measures, disposal, passivation and limitation of the on-ground risk due to re-entry. Some other guidelines for the current debris mitigation are: the avoidance of intentional generation of debris, such as anti-satellite tests, as well as the 25-year deorbit rule for the LEO missions. As it was mentioned in the ESA's latest Space Debris Environment Report, less than 60\% of the LEO satellites comply to these guidelines. \cite{ESA 2019} %(from http://www.esa.int/Safety_Security/Space_Debris/The_cost_of_space_debris)

Even though there are national and international mitigation measures, the compliance is insufficient to stabilize the orbital environment. For this reason, it was investigated that by reducing the residual lifetime from 25 to 5 years we could prevent the increase of inactive population and thus the prevention of collision fragments. The commercial operators should commit to design objects with lower residual life. Thus, the satellite deorbit reliability will be maximized. \cite{Somma 2019}

The aforementioned Space Debris Mitigation Guidelines are applicable not only to mission planning and designing new spacecraft, but also to the existing orbiting objects if possible. The implementation of those guidelines is highly recommended, however it is a voluntarily action due to the fact that they are not legally binding under international law. \cite{UNOOSA}

\bigskip
%SPACE SUSTAINABILITY RATING - HOW WELL AN INDIVIDUAL SATELLITE FOLLOWS THE GUIDELINES.
The fact that the existing international guidelines are not enforceable and the derived standards are not always followed, has led the \textit{World Economic Forum} (WEF) to select a consortium of companies, universities and agencies in order to create a system for rating the sustainability of space systems. \cite{Space sustainability} %Source: https://www.weforum.org/projects/space-sustainability-rating
%It was developed by the World Economic Forum’s Global Future Council on Space. Technologies.
In particular, this project is consisted of an international and multidisciplinary team and the main collaborating entities are the \textit{European Space Agency} (ESA), \textit{Massachusetts Institute of Technology} (MIT), \textit{University of Texas at Austin} and \textit{Bryce Space and Technology}. The so called Space Sustainability Rating (SSR) concept has the goal of encouraging a responsible behavior in space through increasing the transparency of organization's debris mitigation efforts.
%Francesca Letzia and Stijn Lemmens (from ESA) are working for that. Another idea would be also to have the concept of added-value incorportated into that rating. But still it seems very difficult - so it is not part of the parameters.
%The other side of the insurance companies say: If one constellation has a good rating, then the insurer (insurance company) will make a discount. But at the same time this will probably affect the requests for insurance claims on satellites (since the rating is good and thus the risk as well of collision is small - the satellite companies might not want to be insured). *These are just guesses...!*

In order to ensure the long-term sustainability of space, a measurement for every individual satellite or constellation will be carried out, which will show the degree of commitment to the guidelines. The calculation of this score will take into account both quantitative and qualitative parameters. Some determinant parameters are each satellite's design and physical parameters, the economic viability of the satellite's operator (relevant example of \textit{OneWeb} \cite{Oneweb_bankruptcy}, \cite{Cadman}), % In case, a company declares bankruptcy, as OneWeb (UK) did, many already launched satellites are doomed to stay in LEO occupying space in that area and increasing the risk of collision when they will be become non-operational. (There are now in orbit around 70 OneWeb satellites). Chinese investors are interested into buying the company. The United States and other Western nations are not keen on the idea of seeing OneWeb slip into "adversarial hands.")
how trackable the satellite is, estimations concerning potential collisions and the satellite's disposal at the end of its life. \cite{Space sustainability}

Currently there are so many different definitions of space sustainability and this definition needs to be shaped. Nevertheless, this level of sustainability of a mission will be publicly available by placing emphasis on the debris mitigation approach and by giving a positive incentive to satellite operators to increase their responsible behavior and to improve their ratings. An initial beta version of this rating system is going to be released at the end of 2020/ beginning of 2021.
%Source: https://www.weforum.org/projects/space-sustainability-rating

%In a similar way the "Space Traffic Footprint" will be created. It will include data on the size and location of an object, how crowded the orbital area it inhabits already is and if a satellite is maneuverable and trackable.
%(Not necessary to add I think)
%Source: https://phys.org/news/2019-05-space-sustainability-aims-amount-debris.html

%The SSR will give a score related to debris mitigation and alignment with international guidelines. (I said it with other words.)

\bigskip
%ESTIMATION OF ADDED VALUE
As it was mentioned in Chapter \ref{chap:scope}, the goal of this thesis is to give the incentives for the formation of rules which will be based on the added value that a new satellite offers. As an example towards this direction is the report of the independent consultancy, London Economics (LE), which is working in the space sector, regarding the value of the EO capabilities offered by satellites to the UK government. \cite{Value UK} In this case the value is determined based on multiple parameters such as the operational cost saving, the exceptional cost avoidance, better policy decisions and the benefits that the government, the economy and the society gain from the offered EO satellite capabilities. This study can be a great approach and can be serve as an example to other nations and institutes towards prioritizing the missions based on their potential value to the society.

A different way of mitigating the problem of overpopulation, which is also connected to the aforementioned idea of added-value, is the concept of cooperation between entities towards sharing common resources. As it has been also mentioned in the GEO (\textit{Group on Earth Observations}) summit in Tokyo, 2004, when the GEOSS (\textit{Global Earth Observation System of Systems}) was created, in order to have a societal benefit of Earth Observations, data sharing is necessary. \cite{Kramer 2002} %Krmaer p.39
Some cases, which show interest in this direction are the Argentinian \textit{SAOCOM} and the Italian \textit{Cosmos-SkyMed} EO satellites.  %SAR sensors

Another example is the group \textit{BRICS} consisted of the states Brazil, Russia, India, China and South Aftica, and they will join their forces towards a mission regarding natural resources and disaster management. They will use the already existing/ orbiting Sun-Synchornous Remote Sensing satellites. An initiative was also taken from UNOOSA with the name of \hspace{1mm} “Access to Space for All”, which has the goal of connecting and creating necessary conditions for cooperation between the established space actors in order to achieve the SDGs (Sustainable Development Goals).
%Mention link- source: https://www.unoosa.org/oosa/en/ourwork/access2space4all/index.html 

% Cooperation between companies: Planet and Copernicus encourage people to use jointly their data (there is also the challenge “See change, change the world”) --> I don't think that it has the character of mitigating the problem of overpopulation etc. Or that the cooperate for this reason.

\bigskip
\section{LEO: a favorable spot}
\bigskip
%LEO Low Earth Orbit. (Overview - Status?!)

The Low Earth Orbit (LEO) is a favorable spot for placing satellites for various reasons. It's proximity from Earth makes it possible to launch even CubeSats, since their fuel capacity is enough to be able to maintain their position and/or to de-orbit at the end of their mission. Specifically in the recent years more and more small satellites and/or CubeSats are placed in LEO and in large constellations. (Fig. \ref{launch_traffic_LEO}) The need of proliferated constellations is also linked to the low altitude of the LEO region. Namely, for larger coverage, a satellite should be either placed in a higher altitude, or the mission should engage more satellites in order to compensate the smaller field of view that one satellite offers.

LEO is used for remote sensing, Earth observation, human spaceflight, and more. The most crowded one. 
...


%**• Launch traffic into the LEO protected region is changing significantly, fuelled by the proliferation of small
%payloads, i.e. below 10.0 kg in mass, during the last few years in terms of number, but not contributing
%significantly to the mass. (ESA report p.77)


The total number of operating satellites, as of 31.3.2020 including all the launches by that date is 2,666 with 1,918 being in LEO. \cite{UCS} There are some regions of LEO, which have already been declared as critical. More specifically, the three critical regions in LEO are \cite{Kramer 2002}:
\begin{itemize}
\item The first critical region is at the altitude 750-850km, at which the spatial density is increased because of the debris' population. It is characterized as critical, because objects need hundreds of years in order to decay and re-enter the atmosphere.
\item The second critical region is at the altitude 900-1000km, at which there is a high number of massive objects. Drag is also not sufficient in order to help objects decay and thus the spatial density increases over time.
\item The third critical region is at the altitude 1100-1300km. At this region the effect of drag is negligible, which means that any additional object builds up the orbital population.
%The future large constellations will be placed around these altitudes, which makes this region even more alarming. \cite{Somma 2019}
\end{itemize}

%%Some general things about LEO:
%	General altitude range: 160-2000km above Earth’s surface.
%	Orbital periods between 88-127 min.
%	All inclinations possible
%	Orbital velocity (with regard to an observer on Earth): 7-8km/s
%	LEO subclasses/ “shapes”:
%•	Circular LEO orbit is the most common and natural orbit for Earth observation, especially in the altitude range of about 200-900 km with orbital periods of about 90-105min.
%•	Special orbits:
%	Polar orbit
%	Sun-Synchronous orbit (a special case of near-polar orbit!):
%An orbit like this is possible by the fact that the Earth is not a perfect sphere. In the SSO, the daily rotation of the orbital satellite plane (with respect to the equatorial plane) is identical to the mean motion of the fictitious sun around the Earth = which is identical to the mean motion of the Earth around the sun.
%	Check the relationship between the coverage and revisit/ repeat period with the inclination and altitude. (It’s on the application’s part.)

%\bigskip
%\subsection{Critical regions of LEO}
%\bigskip
%
%IADC (Inter-Agency Space Debris Coordination Committee) has declared protected regions \cite{IADC 2007}.

...

\bigskip
\subsection{EO}
\bigskip

%Why EO? It has multiple capabilities
What do EO offers? "Optical imagery, high-revisit, all weather, and nighttime, Synthetic Aperture Radar, radio signal collection satellites that can geolocate signals emissions—essentially offering commercial electronic intelligence capabilities that can support transportation and logistics, emergency search and rescue, or spectrum mapping in addition to its existing applications for national security"
\cite{Hallex}

...